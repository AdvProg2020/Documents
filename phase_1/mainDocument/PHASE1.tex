\documentclass[]{article}
\usepackage{graphicx}
\usepackage[svgnames]{xcolor} 
\usepackage{fancyhdr}

\usepackage[hidelinks]{hyperref}
\usepackage{enumitem}
\usepackage[many]{tcolorbox}
\usepackage{listings }
%\usepackage[a4paper, total={6in, 8in} , top = 2cm,bottom = 4cm]{geometry}
\usepackage[a4paper, total={6in, 8in}]{geometry}
\usepackage{afterpage}
\usepackage{amssymb}
\usepackage{pdflscape}
\usepackage{textcomp}
\usepackage{xecolor}
\usepackage{rotating}
\usepackage[Kashida]{xepersian}
\usepackage[T1]{fontenc}
\usepackage{tikz}
\usepackage[utf8]{inputenc}
\usepackage{PTSerif} 
\usepackage{seqsplit}
\usepackage{changepage}



\usepackage{listings}
\usepackage{xcolor}
\usepackage{sectsty}
 
\definecolor{codegreen}{rgb}{0,0.6,0}
\definecolor{codegray}{rgb}{0.5,0.5,0.5}
\definecolor{codepurple}{rgb}{0.58,0,0.82}
\definecolor{backcolour}{rgb}{0.95,0.95,0.92}
 
\NewDocumentCommand{\codeword}{v}{
\texttt{\textcolor{blue}{#1}}
}
\lstset{language=java,keywordstyle={\bfseries \color{blue}}}

\lstdefinestyle{mystyle}{
    backgroundcolor=\color{backcolour},   
    commentstyle=\color{codegreen},
    keywordstyle=\color{magenta},
    numberstyle=\tiny\color{codegray},
    stringstyle=\color{codepurple},
    basicstyle=\ttfamily\normalsize,
    breakatwhitespace=false,         
    breaklines=true,                 
    captionpos=b,                    
    keepspaces=true,                 
    numbers=left,                    
    numbersep=5pt,                  
    showspaces=false,                
    showstringspaces=false,
    showtabs=false,                  
    tabsize=2
}

\lstset{style=mystyle}

 \settextfont[BoldFont={XB Zar bold.ttf}]{XB Zar.ttf}


\setlatintextfont[Scale=1.0,
 BoldFont={LiberationSerif-Bold.ttf}, 
 ItalicFont={LiberationSerif-Italic.ttf}]{LiberationSerif-Regular.ttf}





\newcommand{\inputsample}[1]{
    ~\\
    \textbf{ورودی نمونه}
    ~\\
    \begin{tcolorbox}[breakable,boxrule=0pt]
        \begin{latin}
            \large{
                #1
            }
        \end{latin}
    \end{tcolorbox}
}

\newcommand{\outputsample}[1]{
    ~\\
    \textbf{خروجی نمونه}

    \begin{tcolorbox}[breakable,boxrule=0pt]
        \begin{latin}
            \large{
                #1
            }
        \end{latin}
    \end{tcolorbox}
}

\newenvironment{changemargin}[2]{%
\begin{list}{}{%
\setlength{\topsep}{0pt}%
\setlength{\leftmargin}{#1}%
\setlength{\rightmargin}{#2}%
\setlength{\listparindent}{\parindent}%
\setlength{\itemindent}{\parindent}%
\setlength{\parsep}{\parskip}%
}%
\item[]}{\end{list}}


\definecolor{foldercolor}{RGB}{124,166,198}
\definecolor{sectionColor}{HTML}{ff5e0e}
\definecolor{subsectionColor}{HTML}{008575}

\definecolor{listColor}{HTML}{00d3b9}

\definecolor{umlrelcolor}{HTML}{3c78d8}


\defpersianfont\titr[Scale=1.5]{Lalezar-Regular.ttf}

\defpersianfont\fehrest[Scale=1.2]{Lalezar-Regular.ttf}

\sectionfont{\color{sectionColor}}  % sets colour of sections



\subsectionfont{\color{subsectionColor}}  % sets colour of sections

\renewcommand{\labelitemii}{$\circ$}


\renewcommand{\baselinestretch}{1.1}
\setlength{\parskip}{1.2pt}

\begin{document}


%%% title pages
\begin{titlepage}
\begin{center}

\textbf{ \Huge{به نام خدا} }
        
\vspace{0.2cm}

\includegraphics[width=0.4\textwidth]{sharif1.png}\\
\vspace{0.2cm}
\textbf{ \Huge{\emph درس برنامه‌سازی پیشرفته} }\\
\vspace{0.25cm}
\textbf{ \Large{ فاز اول پروژه} }
\vspace{0.2cm}
       
 
      \large \textbf{دانشکده مهندسی کامپیوتر}\\\vspace{0.1cm}
    \large   دانشگاه صنعتی شریف\\\vspace{0.2cm}
       \large   ﻧﯿﻢ سال دوم 99-98 \\\vspace{0.10cm}
      \noindent\rule[1ex]{\linewidth}{1pt}
اساتید:\\
    \textbf{{مهدی مصطفی‌زاده، ایمان عیسی‌زاده، امیر ملک‌زاده، علی چکاه}}



    \vspace{0.20cm}

   مهلت ارسال:\\
    \textbf{{12 فروردین - }}
    \textbf{{ساعت 23:59:59}}

    \vspace{0.10cm}
مسئول پروژه:\\
    \textbf{{احمد سلیمی}}
    
        \vspace{0.10cm}
مسئول فاز اول:\\
    \textbf{{تست}}
    
        \vspace{0.10cm}
طراحان فاز اول:\\
    \textbf{{تست}}
    
        \vspace{0.05cm}
مسئول تیم تنظیم داک:\\
    \textbf{{تست}}
    
            \vspace{0.05cm}
اعضای تیم تنظیم داک:\\
    \textbf{{تست}}
\end{center}
\end{titlepage}
%%% title pages


%%% header of pages
\newpage
\pagestyle{fancy}
\fancyhf{}
\fancyfoot{}
\cfoot{\thepage}
\lhead{فاز اول}
\rhead{\includegraphics[width=0.1\textwidth]{sharif.png}\\
دانشکده مهندسی کامپیوتر
}
\chead{پروژه برنامه‌سازی پیشرفته}
%%% header of pages
\renewcommand{\headrulewidth}{2pt}

\KashidaOff


 \Large \textbf{\\\\
}

\section*{{\titr نکات قابل توجه}}

\begin{itemize}
\item
پس از اتمام این فاز، در گیت خود یک تگ با عنوان \lr{"phase\_1"} بزنید. در روز تحویل حضوری این tag بررسی خواهد شد و کدهای پس از آن نمره‌ای نخواهد گرفت.

\item
در روز تحویل حضوری مشارکت تمام اعضای تیم در پروژه بررسی خواهد‌ شد و در صورت عدم مشارکت بعضی از اعضا، نمرهٔ ایشان برای آن فاز پروژه "صفر" لحاظ می‌گردد.

\item
در هر فاز می‌توانید سه روز تاخیر به ازای کسر نمره داشته‌ باشید که به ازای هر روز آن، ۱۰ درصد از نمرهٔ آن فاز را از دست خواهید‌ داد. در مجموع سه‌فاز پروژه، سه روز تاخیر نیز بخشیده خواهد‌ شد.

\item
به ازای هر ساعتی که پروژه را زودتر تحویل دهید، ۱۵ دقیقه به مهلت تاخیر بدون کسر نمره شما اضافه خواهد‌ شد. این مقدار حداکثر یک روز خواهد‌ بود که در صورت ارسال ۴ روز زودتر از ددلاین به شما تعلق خواهد گرفت. \textbf{بنابراین ددلاین‌های پروژه تحت هیچ شرایطی تمدید نخواهد‌ شد.} توصیه می‌شود با برنامه‌ریزی مناسب به ددلاین‌های درس پایبند باشید.

\item
در صورت کشف تقلب از هریک از تیم‌ها، برای بار اول منفی نمرهٔ آن فاز برای آن تیم ثبت می‌شود و برای بار دوم، نمرهٔ منفی کل پروژه برای تیم لحاظ خواهد‌ شد که معادل مردود شدن در درس است.
\end{itemize}

\newpage

\section*{{\titr مقدمه}}

\subsection*{{\titr اهداف پروژه}}

\begin{itemize}

\item
هدف این پروژه، طراحی یک سیستم فروشگاه آنلاین است که در فاز اول آن، صرفا منطق فروشگاه پیاده‌سازی می‌شود.

\item
در این فروشگاه، تعدادی فروشنده، محصولات خود را ارائه داده و خریداران آن‌ها را خریداری می‌کنند. ساختار کلی فروشگاه، شبیه به فروشگاه‌های متداول مانند دیجی‌کالا است.

\item
در این فاز از پروژه، طراحی شی‌ءگرای نرم‌افزار و جداسازی منطق بخش‌های مختلف از یکدیگر به صورت صحیح مورد نظر است.

\end{itemize}

\subsection*{{\titr کلیات پروژه}}

در این فاز، صرفا منطق پروژه، بدون پیاده‌سازی گرافیک یا معماری شبکهٔ آن، باید پیاده‌سازی شود. نحوهٔ ارتباط با کاربر نیز از طریق واسط کاربری کنسول است. توجه داشته باشید که در فاز سوم پروژه، باید سیستم را طبق یک معماری سرور-کلاینت طراحی کنید. نکتهٔ مهم این است که منطق پروژه که در سرور قرار می‌گیرد، باید جدا از واسط کاربری باشد و به راحتی قابل جداسازی باشند. در نتیجه پیشنهاد می‌شود از همین ابتدا نگاهی به این جداسازی داشته‌ باشید و این نکته را در طراحی خود لحاظ کنید.

در ادامهٔ مستند، موجودیت‌ها، نمای کلی رابط کاربری سیستم، نقش‌ها و دستورات لازم شرح داده‌شده است.

\begin{enumerate}[label={نکته \arabic*:}]
\item
 تمامی اطلاعات، اعم از اطلاعات کاربران، محصولات، اطلاعات اصلی فروشگاه مثل دسته‌بندی‌ها و...، باید به صورت خودکار در خارج از برنامه (مثلا روی فایل) ذخیره شوند و پس از \lr{terminate} شدن برنامه و اجرای مجدد آن، بصورت خودکار اطلاعات قبلی خوانده شود و قابل دسترسی باشد. برای این کار می‌توانید از ابزارهای کار با \lr{Json} در جاوا، مثل
  \href{https://www.tutorialspoint.com/gson/gson_quick_guide.htm}{\textcolor{blue}{\lr{Gson}}}
   استفاده‌ کنید.
\newpage

\item
 نحوهٔ تست کردن پروژه در تحویل حضوری بدین صورت خواهد بود که اکانت‌ها با نقش‌های مختلف ساخته‌ می‌شود و در بخش‌های مختلف، بین کاربران جابه‌جا خواهیم‌ شد. برای مثال بایستی پس از \lr{logout} یک کاربر، اطلاعات کاربران دیگر موجود باشد تا با کاربر دیگری بتوان \lr{login} کرد.

\item
در هر جایی از پروژه می‌توانید هرگونه خلاقیتی را به‌کار ببرید. توجه کنید که خواسته‌های واضح پروژه بایستی انجام شوند.
\end{enumerate}




\newpage

\section*{{\titr توضیح بخش‌های مختلف پروژه}}

\subsection*{{\titr بخش‌ها}}

\subsubsection*{{\titr موجودیت‌ها}}

\begin{itemize}

\item
حساب کاربری:

\begin{itemize}
\item
مشخصات فردی:

\begin{itemize}[label=$\blacksquare$]
\item
نام کاربری، نام، نام‌خانوادگی، ایمیل، شماره تلفن، رمز عبور 
\end{itemize}
\item
نقش:

\begin{itemize}[label=$\blacksquare$]
\item
خریدار (یا همان کاربر عادی) ویژگی‌هایی مختص خودش ندارد.	

\item
فروشنده: اسم شرکت/کارگاه/کارخانه

\item
مدیر: ویژگی‌هایی مختص خودش ندارد.

\end{itemize}

\item
لیست کدهای تخفیف شخص

\item
اعتبار

\item
سابقهٔ خرید/فروش

\begin{itemize}[label=$\blacksquare$]
\item
لیستی از لاگ خرید/فروش
\end{itemize}

\end{itemize}


\item
لاگ خرید/فروش:

\begin{itemize}

\item
شناسهٔ لاگ

\item
تاریخ

\item
مبلغ پرداخت شده/مبلغ دریافت شده

\item
مقدار تخفیف اعمال‌شده

\item
لیست محصولات خریده/فروخته شده

\item
نام خریدار

\item
نام فروشنده

\item
وضعیت تحویل

\end{itemize}

\item
حراج:

\begin{itemize}

\item
شناسه‌ی حراج \lr{(offId)}

\item
لیست محصولات

\item
وضعیت حراج (در دست بررسی برای ساخت / در دست بررسی برای ویرایش / تایید شده)

\item
زمان دقیق شروع

\item
زمان دقیق پایان

\item
میزان تخفیف


\begin{itemize}[label = $\blacksquare$]
\item
نکته ۱: این نوع تخفیف را، هر فروشنده روی محصولاتی که خودش ارائه می‌کند می‌تواند اعمال کند. و 
در نتیجهٔ اعمال این تخفیف، قیمت کالا کاهش یافته و مبلغ با اعمال تخفیف به اعتبار فروشنده اضافه می‌شود.

\item
نکته ۲:‌ یک محصول در یک زمان، فقط می‌تواند در لیست یک حراج باشد.
\end{itemize}

\end{itemize}

\item
تخفیف کد دار:

\begin{itemize}
\item
کد تخفیف

\item
زمان دقیق شروع

\item
زمان دقیق پایان

\item
میزان تخفیف (شامل درصد و مقدار حداکثر تخفیف ممکن)

\item
دفعات تکرار تخفیف به‌ازای هر کاربر

\item
لیست کاربران مشمول تخفیف


\begin{itemize}[label = {$\blacksquare$}]
\item
نکته:‌ این تخفیف تنها توسط مدیران، و بر روی همهٔ محصولات اعمال می‌شود، و نه در قیمت کالا بلکه در فاکتور خرید اثر می‌گذارد. یعنی فروشنده همان قیمت بدون تخفیف را دریافت می‌کند.
\end{itemize}

\end{itemize}


\newpage

\item
دسته \lr{(category)}:

\begin{itemize}
\item
اسم

\item
ویژگی‌های مخصوص

\item
زیردسته (امتیازی)

\item
لیست محصولات

\vspace{1cm}


\end{itemize}
\item
کالا:

\begin{itemize}

\item
شناسهٔ کالا \lr{(productId)}

\item
وضعیت کالا (در دست بررسی برای ساخت / در دست بررسی برای ویرایش / تایید شده)

\item
مشخصات عمومی مانند اسم، برند/کمپانی، قیمت، فروشنده، وضعیت موجودی

\item
دسته

\item
توضیحات

\item
میانگین نمره خریداران

\item
لیست نظرات

\begin{itemize}[label = $\blacksquare$]
\item
نکته: به شکل پیش‌فرض هر فروشنده که کالایی اضافه می‌کند، به صورت یک کالای مجزا در نظر گرفته می‌شود؛ ولی اگر سامانه پیاده‌سازی شده از قابلیت چند فروشنده برای یک محصول پشتیبانی کند نمرهٔ امتیازی دارد.
\end{itemize}

\end{itemize}
\newpage
\item
نظر:

\begin{itemize}
\item
کاربر نظر دهنده

\item
کالا

\item
متن نظر

\item
وضعیت نظر (در انتظار تایید/تایید شده/تایید نشده توسط مدیر)

\item
آیا نظر دهنده محصول را خریده است.

\end{itemize}

\item
نمره:

\begin{itemize}

\item
کاربر (تنها کاربری که محصول را خریده)

\item
امتیاز

\item
کالا

\end{itemize}


\end{itemize}


\subsubsection*{{\titr نمای کلی رابط کاربری:}}

\begin{itemize}

\item
صفحهٔ اصلی:

\begin{itemize}

\item
ناحیهٔ کاربری

\item
صفحه محصولات

\item
حراج‌ها

\end{itemize}

\item
ناحیهٔ‌ کاربری:

این بخش مشتمل بر دو حالت است:

\begin{enumerate}

\item
لاگین کرده: در این حالت مشخصات و دسترسی‌های حساب کاربری نشان داده می‌شود.

\item
لاگین نکرده: امکان انتقال به صفحهٔ ثبت‌نام یا لاگین را برای کاربر مهیا می‌کند.
\end{enumerate}

\newpage
\item
پنل ثبت‌نام:

مشخصات فردی و مشخصات مربوط به نقش که در بخش حساب کاربری مطرح شده گرفته می‌شود.

\begin{itemize}[label={$\blacksquare$}]
\item
نکته ۱: ثبت‌نام فقط برای نقش‌های خریدار و فروشنده قابل انجام است.

\item
نکته ۲: ساخت اکانت مدیر تنها توسط مدیر قابل انجام است.

\item
نکته ۳: یک فروشنده هنگام ثبت‌نام نقش فروشنده را درخواست می‌کند و پس از تایید مدیر دسترسی‌های فروشنده را خواهد داشت.

\end{itemize}

\item
صفحه محصولات:

\begin{itemize}
\item
دسته‌‌ها \lr{(category)}

\begin{itemize}[label={$\blacksquare$}]
\item
لیست کالاها

\item
جستجو

\end{itemize}
\end{itemize}

\item
حراج‌ها:

\begin{itemize}
\item
لیست کالاهای حراج‌شده.

\item
جستجو
\end{itemize}

\item
جستجو:

\begin{itemize}
\item
فیلترهای مختص هر دسته (روی ویژگی‌های مختلف دسته)

\begin{itemize}[label={$\blacksquare$}]
\item
می‌تواند بصورت انتخابی (مانند موجود بودن کالا، یا برند کالا) و یا بصورت بازه‌ای (مانند بازه‌قیمت) باشد.
\end{itemize}
\end{itemize}

\end{itemize}

\newpage

\subsection*{{\titr نقش‌ها}}

در این فاز، ۳ نقش مد نظر است.

\begin{itemize}
\item
خریدار:

\begin{itemize}

\item
داده‌های مورد نیاز:

\begin{itemize}[label={$\blacksquare$}]

\item
اطلاعات شخصی حساب کاربری

\item
سبد خرید

\item
سابقه خرید

\item
اعتبار حساب کاربری

\end{itemize}

\item
قابلیت‌ها:

\begin{itemize}[label={$\blacksquare$}]

\item
تغییر اطلاعات شخصی

\item
مشاهدهٔ محصولات

\item
فیلتر کردن و جستجو در میان محصولات

\item
مقایسهٔ دو محصول
\item
خرید

\end{itemize}

\end{itemize}


\item
فروشنده:

\begin{itemize}
\item
اطلاعات شخصی حساب کاربری و اطلاعات مربوط به شرکت

\item
لیست سابقه فروش

\item
تغییر اطلاعات شخصی

\item
لیست محصولات فروشی

\begin{itemize}[label={$\blacksquare$}]
\item
ویرایش هر محصول (شامل ویرایش هر چیزی) که در نهایت به صورت یک درخواست به مدیران ارسال می‌شود.


\item
درخواست حذف یک محصول

\end{itemize}

\item
درخواست افزودن محصول 
\newpage

\item
لیست حراج‌ها

\begin{itemize}[label={$\blacksquare$}]
\item
ویرایش حراج‌ها که بصورت یک درخواست به مدیران ارسال می‌شود.
\end{itemize}


\item
درخواست افزودن یک حراج

\end{itemize}


\item
مدیر:

\begin{itemize}


\item
اطلاعات شخصی حساب کاربری

\item
تغییر اطلاعات شخصی

\item
مشاهده لیست درخواست‌ها

\begin{itemize}[label={$\blacksquare$}]
\item
درخواست‌های ثبت‌نام اکانت فروشنده

\item
درخواست افزودن/ویرایش محصول

\item
درخواست افزودن/ویرایش حراج

\end{itemize}

\item
لیست کد تخفیف‌ها

\begin{itemize}[label={$\blacksquare$}]
\item
ویرایش هر کد تخفیف
\end{itemize}

\item
ایجاد کد تخفیف

\item
مشاهده لیست تمامی کاربران

\begin{itemize}[label={$\blacksquare$}]
\item
امکان حذف یک کاربر
\end{itemize}

\item
امکان افزودن اکانت مدیر

\item
لیست دسته‌ها

\begin{itemize}[label={$\blacksquare$}]
\item
ویرایش یک دسته
\end{itemize}

\item
افزودن دسته
\end{itemize}


\end{itemize}

\newpage

\subsection*{{\titr فرآیندها}}

\begin{itemize}

\item
ثبت‌نام:

\begin{itemize}
\item
وارد کردن اطلاعات شخصی

\item
وارد کردن اطلاعات مربوط به نقش درخواست شده

\end{itemize}

\item
ورود:
\begin{itemize}
\item
وارد کردن نام کاربری و گذرواژه

\end{itemize}

\item
ویرایش اطلاعات کاربری:

\begin{itemize}
\item
از طریق ورود به پنل کاربری، کاربر می‌تواند مشخصات خود (بجز نام کاربری) را تغییر دهد.

\end{itemize}

\item
مشاهدهٔ محصولات، فیلتر کردن و جستجو (بدون ورود به حساب نیز قابل دسترسی است):

روند کار به این صورت است:

\begin{enumerate}

\item
کاربر صفحه محصولات می‌شود.

\item
دستهٔ مورد نظر را انتخاب می‌کند.

\item
فیلتر‌های مورد نظر را وارد می‌کند.

\item
در صورت نیاز جستجو می‌کند یا صفحات بعدی لیست محصولات را بررسی می‌کند. 

\item
امکان رفع فیلترهای اعمال شده نیز می‌بایست وجود داشته باشد.

\end{enumerate}

\item
مقایسه:

\begin{itemize}

\item
کاربر مجاز است دو محصول از یک دسته‌بندی را با یکدیگر مقایسه کند.

\item
پس از اعمال مقایسه، تمامی ویژگی‌های دو محصول در کنار هم می‌آیند. در صورتی که اعداد یا توضیحات یکی از مشخصه‌های یکی از محصولات ناقص بود، خالی نمایش داده می‌شود.

\end{itemize}

\item
افزودن به سبد خرید:

\begin{itemize}

\item
اگر کاربر تا الان وارد سامانه نشده است، وارد صفحهٔ ثبت‌نام/ورود می‌شود.

\begin{itemize}[label = {$\bigstar$}]

\item
قابلیت امتیازی: بدون وارد شدن به حساب بتوان سبد خرید داشت و تنها هنگام پرداخت، ورود به حساب انجام شود.


\end{itemize}

\item
در صورت تایید، محصول وارد سبد خرید می‌شود.
\end{itemize}

\item
سبد خرید

\begin{itemize}

\item
مشاهده سبد خرید

\item
در صورت تایید، وارد صفحهٔ مشخصات دریافت‌کننده \lr{(shipping)} می‌شود.

\item
در صورت تایید، وارد صفحهٔ ثبت کد تخفیف و تایید نهایی خرید می‌شود. (استفاده از کد تخفیف برای هر فاکتور تنها یکبار قابل استفاده است)

\item
تایید کسر مبلغ از کیف پول (یا نبود موجودی کافی)

\item
بازگشت به صفحهٔ اصلی

\end{itemize}

\item
فروش:

\begin{itemize}

\item
از طریق پنل کاربری کاربر فروشنده قابل انجام است.

\item
هر فروشنده محصول خود را به عنوان کالای جدید معرفی می‌کند.

\begin{itemize}[label={$\blacksquare$}]
\item
نکته:‌ در صورتی که ویژگی چند فروشنده برای یک محصول پشتیبانی می‌شود، فروشنده درخواست خود برای فروشنده شدن یک کالای موجود ثبت می‌کند.
\end{itemize}

\item
اعلام کردن قیمت، تعداد موجودی، توضیحات و ویژگی‌های آن کالا

\end{itemize}
\item
ویرایش اطلاعات یک محصول:

\begin{itemize}

\item
از طریق پنل کاربری کاربر فروشنده/مدیر قابل انجام است.

\item
درخواست ویرایش اطلاعات کالای خود را پس از ثبت مشخصات جدید ثبت می‌کند.
\end{itemize}

\newpage

\item
ایجاد دسته‌بندی جدید:

\begin{itemize}

\item
این فرایند تنها از طریق پنل کاربری مدیر قابل انجام است.

\item
نام و ویژگی‌های آن دسته را تعیین می‌کند.

\end{itemize}
\item
ایجاد تخفیف کد دار:

\begin{itemize}

\item
این فرایند تنها از طریق پنل کاربری مدیر قابل انجام است.

\item
تمام مشخصات تخفیف را تعیین می‌کند.

\item[{$\bigstar$}]
قابلیت امتیازی: بر اساس رویداد خاصی (مثلا مجموع خرید بیشتر از ۱ میلیون تومان شود) به شخص کد تخفیف داده‌ شود.

\item[{$\bigstar$}]
قابلیت امتیازی: به صورت دوره‌ای به تعدادی کاربر اتفاقی کد تخفیف داده‌ شود.

\end{itemize}

\item
ویرایش تخفیف کد دار:

\begin{itemize}

\item
این فرایند تنها از طریق پنل کاربری مدیر قابل انجام است.

\item
مشخصات تخفیف (بجز کد) را تغییر می‌دهد.

\end{itemize}

\item
ایجاد حراج:

\begin{itemize}
\item
این فرایند توسط فروشنده قابل انجام است.

\item
لیست محصولات مورد تخفیف، درصد و سقف مقدار تخفیف و ددلاین را مشخص می‌کند.

\item
درخواست را به مدیر ارسال می‌کند.


\end{itemize}
\item
ویرایش حراج:


\begin{itemize}

\item
این فرایند توسط فروشنده قابل انجام است.

\item
می‌تواند اطلاعات مربوط به یک حراج خود را تغییر دهد.

\item
درخواست را به مدیر ارسال می‌کند.

\end{itemize}

\newpage

\item
نمره دهی:


\begin{itemize}

\item
کاربر وارد ناحیه کاربری می‌شود.

\item
سابقهٔ خرید خود را مشاهده می‌کند و به محصول مورد نظر نمره می‌دهد.

\begin{itemize}[label = {$\blacksquare$}]

\item
نکته: واضح است که محصول باید توسط کاربر خریده شده باشد تا بتواند به آن امتیاز دهد.

\end{itemize}



\end{itemize}
\item
ثبت نظر:

\begin{itemize}

\item
وارد صفحهٔ محصول مورد نظر می‌شود.

\item
نظر خود را درمورد آن محصول می‌نویسد.

\item
سامانه به صورت خودکار تعیین می‌کند که نظر دهنده کیست و آیا این محصول را خریده است یا خیر.

\end{itemize}

\end{itemize}


\newpage

\section*{{\titr دستورات مورد نیاز}}

\subsection*{{\titr نکات مهم:}}

\begin{itemize}[label={$\bigstar$}]

\item
در این بخش، دستورات مورد نیاز در هر صفحه آمده‌ است. توجه کنید که دستورات به صورت کلی توضیح داده‌شده‌اند و هر جا به جزئیات اشاره نشده‌ است، می‌توانید دلخواه عمل کنید.

\item
در همه منوها باید امکان لاگین و رجیستر برای شخص لاگین‌نشده و امکان لاگ اوت برای شخص لاگین شده وجود داشته باشد و پس از اتمام فرایند به صفحه‌ای که قبلا بوده برگردد.

\item
در همه منوها یک دستور \lr{help} موجود باشد که لیست دستورات آن منو را نشان دهد.

\item
در همه منوها هر جا لیستی نمایش داده می‌شود، بتوان از طریق دستور \lr{sort by [field]} نمایش لیست را تغییر داد.

\item
در هر یک از زیر منوهای برنامه با دستور \lr{back} به منوی قبلی برمی‌گردد.



\end{itemize}



\end{document}









