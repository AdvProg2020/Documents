\documentclass[]{article}
\usepackage{graphicx}
\usepackage[svgnames]{xcolor} 
\usepackage{fancyhdr}
\usepackage{tocloft}
\usepackage[hidelinks]{hyperref}
\usepackage{enumitem}
\usepackage[many]{tcolorbox}
\usepackage{listings }
%\usepackage[a4paper, total={6in, 8in} , top = 2cm,bottom = 4cm]{geometry}
\usepackage[a4paper, total={6in, 8in} , top = 2cm,bottom = 4cm]{geometry}
\usepackage{afterpage}
\usepackage{amssymb}
\usepackage{pdflscape}
\usepackage{textcomp}
\usepackage{xecolor}
\usepackage{rotating}
\usepackage[Kashida]{xepersian}
\usepackage[T1]{fontenc}
\usepackage{tikz}
\usepackage[utf8]{inputenc}
\usepackage{PTSerif} 
\usepackage{seqsplit}
\usepackage{changepage}


\usepackage{listings}
\usepackage{xcolor}
\usepackage{sectsty}

\setcounter{secnumdepth}{0}
 
\definecolor{codegreen}{rgb}{0,0.6,0}
\definecolor{codegray}{rgb}{0.5,0.5,0.5}
\definecolor{codepurple}{rgb}{0.58,0,0.82}
\definecolor{backcolour}{rgb}{0.95,0.95,0.92}
\definecolor{blanchedalmond}{rgb}{1.0, 0.92, 0.8}
\definecolor{brilliantlavender}{rgb}{0.96, 0.73, 1.0}
 
\NewDocumentCommand{\codeword}{v}{
\texttt{\textcolor{blue}{#1}}
}
\lstset{language=java,keywordstyle={\bfseries \color{blue}}}

\lstdefinestyle{mystyle}{
    backgroundcolor=\color{backcolour},   
    commentstyle=\color{codegreen},
    keywordstyle=\color{magenta},
    numberstyle=\tiny\color{codegray},
    stringstyle=\color{codepurple},
    basicstyle=\ttfamily\normalsize,
    breakatwhitespace=false,         
    breaklines=true,                 
    captionpos=b,                    
    keepspaces=true,                 
    numbers=left,                    
    numbersep=5pt,                  
    showspaces=false,                
    showstringspaces=false,
    showtabs=false,                  
    tabsize=2
}

\lstset{style=mystyle}

 \settextfont[BoldFont={XB Zar bold.ttf}]{XB Zar.ttf}


\setlatintextfont[Scale=1.0,
 BoldFont={LiberationSerif-Bold.ttf}, 
 ItalicFont={LiberationSerif-Italic.ttf}]{LiberationSerif-Regular.ttf}





\newcommand{\inputsample}[1]{
    ~\\
    \textbf{ورودی نمونه}
    ~\\
    \begin{tcolorbox}[breakable,boxrule=0pt]
        \begin{latin}
            \large{
                #1
            }
        \end{latin}
    \end{tcolorbox}
}

\newcommand{\outputsample}[1]{
    ~\\
    \textbf{خروجی نمونه}

    \begin{tcolorbox}[breakable,boxrule=0pt]
        \begin{latin}
            \large{
                #1
            }
        \end{latin}
    \end{tcolorbox}
}

\newtcolorbox{mybox}[2][]{colback=red!5!white,
colframe=red!75!black,fonttitle=\bfseries,
colbacktitle=red!85!black,enhanced,
attach boxed title to top center={yshift=-2mm},
title=#2,#1}

\newenvironment{changemargin}[2]{%
\begin{list}{}{%
\setlength{\topsep}{0pt}%
\setlength{\leftmargin}{#1}%
\setlength{\rightmargin}{#2}%
\setlength{\listparindent}{\parindent}%
\setlength{\itemindent}{\parindent}%
\setlength{\parsep}{\parskip}%
}%
\item[]}{\end{list}}


\definecolor{foldercolor}{RGB}{124,166,198}
\definecolor{sectionColor}{HTML}{ff5e0e}
\definecolor{subsectionColor}{HTML}{008575}

\definecolor{listColor}{HTML}{00d3b9}

\definecolor{umlrelcolor}{HTML}{3c78d8}

\definecolor{subsubsectionColor}{HTML}{3c78d8}

\defpersianfont\authorFont[Scale=0.9]{XB Zar bold.ttf}


\defpersianfont\titr[Scale=1.5]{Lalezar-Regular.ttf}

\defpersianfont\fehrest[Scale=1.2]{Lalezar-Regular.ttf}

\defpersianfont\fehrestTitle[Scale=3.0]{Lalezar-Regular.ttf}

\defpersianfont\fehrestContent[Scale=1.2]{XB Zar bold.ttf}


\sectionfont{\color{sectionColor}}  % sets colour of sections
\subsectionfont{\color{subsectionColor}}  % sets colour of sections
\subsubsectionfont{\color{subsubsectionColor}}


\renewcommand{\labelitemii}{$\circ$}


\renewcommand{\baselinestretch}{1.1}


\renewcommand{\contentsname}{فهرست}

\renewcommand{\cfttoctitlefont}{\fehrestTitle}


\renewcommand\cftsecfont{\color{sectionColor}\fehrestContent\selectfont}
\renewcommand\cftsubsecfont{\color{subsectionColor}\fehrestContent\selectfont}
\renewcommand\cftsubsubsecfont{\color{subsubsectionColor}\fehrestContent\selectfont}
%\renewcommand{\cftsecpagefont}{\color{sectionColor}}

\setlength{\parskip}{1.2pt}

\begin{document}


%%% title pages
\begin{titlepage}
\begin{center}

\textbf{ \Huge{به نام خدا} }
        
\vspace{0.2cm}

\includegraphics[width=0.4\textwidth]{sharif1.png}\\
\vspace{0.2cm}
\textbf{ \Huge{\emph درس برنامه‌سازی پیشرفته} }\\
\vspace{0.25cm}
\textbf{ \Large{ فاز دوم پروژه} }
\vspace{0.2cm}
       
 
      \large \textbf{دانشکده مهندسی کامپیوتر}\\\vspace{0.1cm}
    \large   دانشگاه صنعتی شریف\\\vspace{0.2cm}
       \large   ﻧﯿﻢ سال دوم 99-98 \\\vspace{0.10cm}
      \noindent\rule[1ex]{\linewidth}{1pt}
اساتید:\\
    \textbf{{مهدی مصطفی‌زاده، ایمان عیسی‌زاده، امیر ملک‌زاده، علی چکاه}}



    \vspace{0.20cm}

   مهلت ارسال:\\
    \textbf{{۲۳ اردیبهشت - }}
    \textbf{{ساعت 23:59:59}}

    \vspace{0.10cm}
مسئول پروژه:\\
    \textbf{\authorFont{احمد سلیمی}}
    
        \vspace{0.10cm}
مسئولین فاز دوم:\\
    \textbf{\authorFont{سید مهدی فقیه و زﻫﺮا یوسفی جمارانی}}
    
        \vspace{0.10cm}
طراحان فاز دوم:\\
    \textbf{\authorFont{}}
    
        \vspace{0.05cm}
مسئول تنظیم داک:\\
    \textbf{\authorFont{امیرمهدی نامجو}}
    

\end{center}
\end{titlepage}
%%% title pages


%%% header of pages
\newpage
\pagestyle{fancy}
\fancyhf{}
\fancyfoot{}
\cfoot{\thepage}
\lhead{فاز دوم}
\rhead{\includegraphics[width=0.1\textwidth]{sharif.png}\\
دانشکده مهندسی کامپیوتر
}
\chead{پروژه برنامه‌سازی پیشرفته}
%%% header of pages
\renewcommand{\headrulewidth}{2pt}

\KashidaOff



\tableofcontents

\newpage

 \Large \textbf{\\
}


\section*{{\titr توضیحات کلی}}
\addcontentsline{toc}{section}{{\fehrestContent توضیحات کلی}}

\textbf{\textcolor{red}{توجه بسیار مهم:}}
حتما داک نمرات این فاز را به دقت بررسی نمایید؛ پیاده سازی شما باید شامل آن موارد باشد تا نمره هر کدام را بگیرید. هرگونه پیاده‌سازی یا طراحی اضافه‌تر، کاملاً اختیاری، و فاقد نمره‌ی اضافه می‌باشد. در این داک فقط کلیت پیاده سازی ذکر شده است.

در این فاز شما باید گرافیکی برای لاجیک خود پیاده سازی کنید. توجه به نکات و راهنمایی‌های زیر شما را در پیاده سازی این فاز یاری می‌دهد:


\begin{itemize}
\item
نباید پیاده سازی لزوما به همان شکلی که در عکس‌ها می‌بینید انجام شود؛ ولی توجه کنید حتما باید بخش‌های الزامی را پیاده سازی کنید.

\item
عکس‌هایی که در قسمت‌های مختلف آورده شده است صرفا برای ایده گرفتن است و لزومی به پیاده سازی همانند آن نیست.


\item
\textcolor{red}{باید}
 از فونت‌های مناسب استفاده کنید؛ میتوانید از
   \href{https://www.kenney.nl/assets/kenney-fonts}{\textcolor{blue}{\underline{\lr{fonts}}}}
    و
     \href{https://www.behance.net/collection/4860923/Free-Fonts}{\textcolor{blue}{\underline{\lr{free-font }}}}
    استفاده کنید.

\item

می‌توانید از سایت‌های
 \href{https://www.flaticon.com/}{\textcolor{blue}{\underline{\lr{flaticon}}}} 
 - 
 \href{https://icons8.com/}{\textcolor{blue}{\underline{\lr{icons8}}}}
  -
   \href{https://www.iconninja.com/}{\textcolor{blue}{\underline{\lr{icon ninja}}}}
   آیکون‌های مورد نیاز خود را پیدا کنید.

\item

می‌توانید با استفاده از مستطیل و شفاف سازی آن‌ها مانند
 (\href{https://raw.githubusercontent.com/titansarus/Documents/master/phase_2/main/images/img1.jpg}{\textcolor{blue}{\underline{{این عکس}}}})  و یا با استفاده از عکس‌های مختلف  دکمه‌های مورد نیاز خود را بسازید. همچنین در 
  (
\href{https://hannemann.itch.io/ui-button-pack-free}{\textcolor{blue}{\underline{\lr{button-pack}}}}
 - \href{https://www.vecteezy.com/vector-art/116983-digital-game-button}{\textcolor{blue}{\underline{\lr{round-button}}}}
  -
  \href{https://www.clickminded.com/button-generator/}{\textcolor{blue}{\underline{\lr{button-factory}}}} -
   \href{https://pngtree.com/free-png-vectors/hexagon}{\textcolor{blue}{\underline{\lr{hexagon}}}}
   ) می‌توانید انواع دکمه‌ها را مشاهده کنید.


\item

می‌توانید برای کل قسمت‌ها یا برای هر قسمت متناسب با آن، عکسی به عنوان background قرار دهید (و یا رنگ background را تغییر دهید) البته این مورد به جز
\textcolor{red}{صفحهٔ اصلی}
  برای دیگر صفحات اجباری نیست و نمره اضافه‌ای نیز ندارد. لینک‌های مرتبط:

\href{https://pngtree.com/free-backgrounds}{\textcolor{blue}{\underline{\lr{background}}}} 
-
 \href{https://raw.githubusercontent.com/titansarus/Documents/master/phase_2/main/images/img2.jpg}{\textcolor{blue}{\underline{\lr{background 2 }}}}
 -
  \href{https://github.com/titansarus/Documents/blob/master/phase_2/main/images/img3.jpg}{\textcolor{blue}{\underline{\lr{sale-background }}}} 


\item

ممکن است این آیکون‌ها به شما کمک کنند:

\href{https://gamedeveloperstudio.itch.io/ui-icons}{\textcolor{blue}{\underline{\lr{icons}}}}
 -
\href{https://www.kenney.nl/assets/game-icons}{\textcolor{blue}{\underline{\lr{icons}}}}
   -
\href{https://www.kenney.nl/assets/ui-pack}{\textcolor{blue}{\underline{\lr{ui\_icons}}}}
     -
      \href{http://vecteezy.com/vector-art/112447-preloader-ui-progress}{\textcolor{blue}{\underline{\lr{progress-icon}}}}
       -
       \href{https://www.vecteezy.com/vector-art/144976-set-of-coupon-sale-vectors}{\textcolor{blue}{\underline{\lr{sale-icon}}}}
       -
        \href{https://www.iconninja.com/tag/offer-icon}{\textcolor{blue}{\underline{\lr{offer-icon}}}}
         - 
         \href{https://www.iconninja.com/tag/add-icon}{\textcolor{blue}{\underline{\lr{buy \& add}}}}

\item

در هر بخش اگر عمل مورد نظر موفق نبود، باید خطای مناسب را نمایش دهید؛ توجه کنید که باید این خطاها حتما به صورت گرافیکی نمایش داده شوند ولی اجباری برای اینکه به چه شکلی باشند وجود ندارد و می‌توانید از عکس یا text یا … استفاده کنید.

می توانید از عکس‌های زیر ایده بگیرید:

\begin{center}
\includegraphics[width=0.7\textwidth]{images/image4.png}
\end{center}

\begin{center}
\includegraphics[width=0.7\textwidth]{images/image5.png}
\end{center}

\end{itemize}



\newpage

\section*{{\titr باید‌های پیاده‌سازی}}
\addcontentsline{toc}{section}{{\fehrestContent باید‌های پیاده‌سازی}}

\textbf{\textcolor{red}{توجه:}}
اگر زمانی که برنامه را اجرا می‌کنید، هنوز هیچ اکانت مدیری ساخته نشده است، باید ابتدا مشخصات اکانت مدیر دریافت شود و سپس وارد صفحهٔ اصلی برنامه شوید. در غیر این‌صورت(یعنی اگر مدیر وجود داشت) با اجرا کردن برنامه وارد صفحهٔ اصلی می شوید.

\subsection*{{\titr صفحهٔ اصلی}}

\addcontentsline{toc}{subsection}{{\fehrestContent صفحهٔ اصلی}}

شما باید یک منو برای انتخاب قسمت‌های مختلف یعنی ناحیه کاربری، محصولات و حراج‌ها داشته باشید.

\textbf{\textcolor{red}{توجه:}}
 دقت داشته باشید که برای این scene باید الزاما یکی از دو مورد زیر را نیز انجام دهید:

\begin{itemize}
\item
تغییر رنگ background

\item
قرار دادن عکسی به عنوان background

میتوانید از عکس‌های زیر ایده بگیرید:

\begin{center}
\includegraphics[width=0.3\textwidth]{images/image6.png}
\end{center}

\begin{center}
\includegraphics[width=0.6\textwidth]{images/image7.png}
\end{center}
\end{itemize}

\newpage

\subsection*{{\titr ناحیه کاربری}}
\addcontentsline{toc}{subsection}{{\fehrestContent ناحیه کاربری}}

 شما باید در این قسمت گرافیک تمام ناحیه کاربری که در فاز اول پیاده سازی کردید را پیاده سازی کنید.
 
اگر کاربر لاگین کرده باشد مشخصات و دسترسی‌های حساب کاربری نشان داده می‌شود؛ در غیر این صورت  امکان ثبت‌نام یا ورود برای کاربر وجود داشته باشد.

\textbf{\textcolor{red}{توجه:}}
 امکان دسترسی به ناحیه کاربری از تمام صفحات دیگر باید وجود داشته باشد.



\subsubsection*{{\titr پنل ثبت‌نام و ورود}}
\addcontentsline{toc}{subsubsection}{{\fehrestContent پنل ثبت‌نام و ورود}}

شما باید برای ثبت‌نام و همچنین ورود هر کاربر صفحه‌ای را پیاده سازی کنید. در صفحه‌ٔ‌ ثبت نام، باید برای دو نوع کاربر (خریدار و فروشنده)، فرم مربوط به آن موجود باشد و فیلدهای مربوط به هرکدام به کاربر نمایش داده‌ شود. توجه کنید که ثبت نام خریدار در صورت نبود مشکل بلافاصله انجام می‌پذیرد و بعد از آن می‌تواند login کند اما برای فروشنده، باید ابتدا ثبت نام آن توسط مدیر تایید شود و سپس می‌تواند login کند. پس از ورود نیز به همان صفحه‌ای که قبل از آن بوده می‌رود.


می‌توانید از عکس‌های زیر ایده بگیرید:






\begin{center}
\includegraphics[width=0.9\textwidth]{images/image9.png}
\end{center}


\begin{center}
\includegraphics[width=0.9\textwidth]{images/image10.png}
\end{center}


\begin{center}
\includegraphics[width=0.9\textwidth]{images/image11.png}
\end{center}

\begin{center}
\includegraphics[width=0.6\textwidth]{images/image12.png}
\end{center}

\begin{center}
\includegraphics[width=0.6\textwidth]{images/image8.png}
\end{center}


\newpage


\subsubsection*{{\titr حساب کاربری}}
\addcontentsline{toc}{subsubsection}{{\fehrestContent حساب کاربری}}

شما باید صفحه‌ای برای نشان دادن اطلاعات کاربر داشته باشید؛ به عنوان مثال برای خریدار نمایش اطلاعات زیر الزامی است:

\begin{itemize}
\item
اطلاعات شخصی نظیر نام کاربری، نام، نام‌خانوادگی، ایمیل، شماره، رمز عبور

\item
نقش فرد (اگر فروشنده است اسم شرکت/کارگاه/کارخانه نیز ذکر شود.)

\item
دکمه انتقال به سبد خرید

\item
اعتبار

\item
دکمه‌ انتقال به صفحه سابقهٔ خرید

\item
لیست کدهای تخفیف شخص




\end{itemize}


برای مدیر و فروشنده نیز بایستی تمامی موارد گفته شده در فاز 1 را پیاده کنید.


می‌توانید از عکس‌های زیر ایده بگیرید:


\begin{center}
\includegraphics[width=0.9\textwidth]{images/image13.png}
\end{center}


\begin{center}
\includegraphics[width=0.9\textwidth]{images/image14.png}
\end{center}


\begin{center}
\includegraphics[width=0.9\textwidth]{images/image15.png}
\end{center}

\newpage

\subsection*{{\titr محصولات}}
\addcontentsline{toc}{subsection}{{\fehrestContent محصولات}}

\subsubsection*{{\titr صفحهٔ محصولات}}
\addcontentsline{toc}{subsubsection}{{\fehrestContent صفحهٔ محصولات}}


در این صفحه، باید لیست دسته‌بندی‌ها، ابزار sort، ابزار فیلتر و لیست محصولات طبق فیلتر اعمال شده و با ترتیبی که مشخص شده است نمایش داده شود. در حالت پیش‌فرض، هیچ فیلتری اعمال نشده است و همه‌ی محصولات به ترتیب تاریخ افزودن محصول نمایش داده می‌شوند.


\textbf{\textcolor{red}{توجه ۱:}}
برای اعمال فیلتر، کافیست یک دکمهٔ وجود داشته باشد که با زدن آن، لیست محصولات با توجه به فیلتر update شود.


\textbf{\textcolor{red}{توجه ۲:}}
 هر محصول در لیست محصولات باید دارای عکس، عنوان، قیمت و نمره باشد.


می‌توانید از عکس‌های زیر ایده بگیرید:


\begin{center}
\includegraphics[width=0.9\textwidth]{images/image16.png}
\end{center}


\begin{center}
\includegraphics[width=0.9\textwidth]{images/image19.png}
\end{center}



\begin{center}
\includegraphics[width=0.7\textwidth]{images/image17.png}
\end{center}



\begin{center}
\includegraphics[width=0.7\textwidth]{images/image18.png}
\end{center}







\begin{center}
\includegraphics[width=0.9\textwidth]{images/image20.png}
\end{center}


\begin{center}
\includegraphics[width=0.9\textwidth]{images/image21.png}
\end{center}

\newpage
\subsubsection*{{\titr کالا}}
\addcontentsline{toc}{subsubsection}{{\fehrestContent کالا}}

باید صفحه‌ای داشته باشید که برای هر کالا هنگام کلیک بر روی آن، مشخصات آن کالا را نشان دهد. مواردی که باید حتما پیاده سازی شوند: 
\begin{itemize}
\item
ویژگی‌های عمومی (نام، توضیحات، قیمت، نمره و...) 

\item
ویژگی‌های خاص دسته‌بندی محصول

\item
 لیست نظرات (هر نظر شامل نام کاربری نظر دهنده، متن نظر و این که کالا را خریده یا نه است)
 
 \item
امکان نظر دادن

\item
نمره دادن به محصول (در صورتی که آن را خریده باشد)

\item 
دکمه‌ی افزودن به سبد خرید

\item
اگر از چند فروشنده پشتیبانی می‌کنید، لیست فروشندگان و قابلیت انتخاب فروشنده باید وجود داشته باشد.

\end{itemize}

می‌توانید از عکس‌های زیر ایده بگیرید:


\begin{center}
\includegraphics[width=0.7\textwidth]{images/image22.png}
\end{center}


\begin{center}
\includegraphics[width=1.0\textwidth]{images/image23.png}
\end{center}



\begin{center}
\includegraphics[width=1.0\textwidth]{images/image24.png}
\end{center}


\begin{center}
\includegraphics[width=0.9\textwidth]{images/image25.png}
\end{center}


\begin{center}
\includegraphics[width=0.9\textwidth]{images/image26.png}
\end{center}

\newpage

\subsection*{{\titr حراج‌ها}}
\addcontentsline{toc}{subsection}{{\fehrestContent حراج‌ها}}

شما باید صفحه‌ای برای نشان دادن حراج‌ها داشته باشید. ساختار این صفحه کاملا مشابه صفحهٔ محصولات است، با این تفاوت که فقط کالاهای دارای حراج را نمایش می‌دهد.

\textbf{\textcolor{red}{توجه ۱:}}
این صفحه را میتوانید به عنوان یک فیلتر (مثلا فیلتر تخفیف) در صفحه محصولات پیاده کنید. (در این صورت نیازی به داشتن دکمه ای برای انتقال به صفحهٔ حراج‌ها در منوی اصلی نیست)


\textbf{\textcolor{red}{توجه 2:}}
هر محصول در این لیست باید علاوه بر عکس، عنوان، قیمت و نمره،زمان باقی‌مانده به پایان حراج و میزان تخفیف را نیز شامل شود.


\subsection*{{\titr سبد خرید}}
\addcontentsline{toc}{subsection}{{\fehrestContent سبد خرید}}

شما باید صفحه‌ای برای نشان دادن سبد خرید کاربر داشته باشید. مواردی که باید حتما پیاده سازی شوند:

\begin{itemize}
\item
کالا‌های موجود در سبد خرید

\item
قیمت و تعداد کالاها

\item
هزینه نهایی اجناس موجود در سبد کالا

\item
گزینه‌ای برای افزایش یا کاهش تعداد کالای موجود در سبد کالا

\item
گزینه‌ای برای انتقال به صفحه پرداخت (در ادامه توضیح داده شده است)


\end{itemize}

\textbf{\textcolor{red}{توجه:}}
 در صورت کلیک بر روی هر محصول باید وارد صفحه مربوط به آن کالا شوید.
 
 
می‌توانید از عکس های زیر ایده بگیرید:


\begin{center}
\includegraphics[width=0.9\textwidth]{images/image28.png}
\end{center}


\begin{center}
\includegraphics[width=1.1\textwidth]{images/image27.png}
\end{center}

\newpage


\subsection*{{\titr صفحه پرداخت }}
\addcontentsline{toc}{subsection}{{\fehrestContent صفحه پرداخت }}

در این صفحه باید مراحل زیر را به ترتیب پیاده کنید:

\begin{enumerate}
\item
اطلاعات دریافت‌کننده (آدرس، تلفن و...)

\item
دریافت کد تخفیف

\item
دکمهٔ پرداخت

\end{enumerate}

پس از زدن دکمهٔ پرداخت، گزارش نهایی خرید نمایش داده می‌شود. (اگر موفق بود تایید کسر مبلغ از اعتبار حساب، در غیر این صورت پیغام خطای مناسب نمایش داده شود.)

\textbf{\textcolor{red}{توجه:}}
 دقت کنید که مراحل بالا الزاما باید به ترتیب پیاده شوند و مثلا نباید وقتی که هنوز اطلاعات دریافت‌کننده وارد نشده است، دکمهٔ پرداخت نشان داده شود.


\subsection*{{\titr سابقهٔ خرید/فروش }}
\addcontentsline{toc}{subsection}{{\fehrestContent سابقهٔ خرید/فروش }}

در این صفحه بایستی لیست لاگ‌های خرید یا فروش کاربر را نمایش دهید و برای هر لاگ نیز تمام مشخصات آن -که در فاز 1 پیاده کردید- را نمایش دهید.

توجه کنید که می‌توانید این صفحه را با صفحه حساب کاربری ترکیب کنید ولی حتما بایستی این موارد پیاده شده باشند.


\section*{{\titr توضیح بخش‌های مختلف پروژه}}
\addcontentsline{toc}{section}{{\fehrestContent توضیح بخش‌های مختلف پروژه}}

\subsection*{{\titr بخش‌ها}}
\addcontentsline{toc}{subsection}{{\fehrestContent بخش‌ها}}

\subsubsection*{{\titr موجودیت‌ها}}
\addcontentsline{toc}{subsubsection}{{\fehrestContent موجودیت‌ها}}

\begin{itemize}

\item
حساب کاربری:

\begin{itemize}
\item
مشخصات فردی:

\begin{itemize}[label=$\blacksquare$]
\item
نام کاربری، نام، نام‌خانوادگی، ایمیل، شماره تلفن، رمز عبور 
\end{itemize}
\item
نقش:

\begin{itemize}[label=$\blacksquare$]
\item
خریدار (یا همان کاربر عادی) ویژگی‌هایی مختص خودش ندارد.	

\item
فروشنده: اسم شرکت/کارگاه/کارخانه

\item
مدیر: ویژگی‌هایی مختص خودش ندارد.

\end{itemize}

\item
لیست کدهای تخفیف شخص

\item
اعتبار

\item
سابقهٔ خرید/فروش

\begin{itemize}[label=$\blacksquare$]
\item
لیستی از لاگ خرید/فروش
\end{itemize}

\end{itemize}


\item
لاگ خرید:

\begin{itemize}

\item
شناسهٔ لاگ

\item
تاریخ

\item
مبلغ پرداخت شده

\item
مقدار تخفیف اعمال‌شده (تخفیف کددار)

\item
لیست محصولات خریده


\item
نام فروشنده

\item
وضعیت تحویل

\end{itemize}

\newpage

\item
لاگ فروش:

\begin{itemize}

\item
شناسهٔ لاگ

\item
تاریخ

\item
مبلغ دریافت شده

\item
مبلغ کسر شده بابت حراج (در صورت وجود)

\item
محصول فروخته شده(پیاده‌سازی این قسمت به صورت لیست مانعی ندارد)

\item
نام خریدار

\item
وضعیت ارسال


\end{itemize}


\item
حراج:

\begin{itemize}

\item
شناسه‌ی حراج \lr{(offId)}

\item
لیست محصولات

\item
وضعیت حراج (در دست بررسی برای ساخت / در دست بررسی برای ویرایش / تایید شده)

\item
زمان دقیق شروع

\item
زمان دقیق پایان

\item
میزان تخفیف


\begin{itemize}[label = $\blacksquare$]
\item
نکته ۱: این نوع تخفیف را، هر فروشنده روی محصولاتی که خودش ارائه می‌کند می‌تواند اعمال کند. و 
در نتیجهٔ اعمال این تخفیف، قیمت کالا کاهش یافته و مبلغ با اعمال تخفیف به اعتبار فروشنده اضافه می‌شود.

\item
نکته ۲:‌ یک محصول در یک زمان، فقط می‌تواند در لیست یک حراج باشد.
\end{itemize}

\end{itemize}

\item
تخفیف کد دار:

\begin{itemize}
\item
کد تخفیف

\item
زمان دقیق شروع

\item
زمان دقیق پایان

\item
میزان تخفیف (شامل درصد و مقدار حداکثر تخفیف ممکن)

\item
دفعات تکرار تخفیف به‌ازای هر کاربر

\item
لیست کاربران مشمول تخفیف


\begin{itemize}[label = {$\blacksquare$}]
\item
نکته:‌ این تخفیف تنها توسط مدیران، و بر روی همهٔ محصولات اعمال می‌شود، و نه در قیمت کالا بلکه در فاکتور خرید اثر می‌گذارد. یعنی فروشنده همان قیمت بدون تخفیف را دریافت می‌کند.
\end{itemize}

\end{itemize}



\item
دسته \lr{(category)}:

\begin{itemize}
\item
اسم

\item
ویژگی‌های مخصوص

\item
زیردسته (امتیازی)

\item
لیست محصولات




\end{itemize}
\item
کالا:

\begin{itemize}

\item
شناسهٔ کالا \lr{(productId)}

\item
وضعیت کالا (در دست بررسی برای ساخت / در دست بررسی برای ویرایش / تایید شده)

\item
مشخصات عمومی مانند اسم، برند/کمپانی، قیمت، فروشنده، وضعیت موجودی

\item
دسته

\item
مشخصات خاص دسته

\item
توضیحات

\item
میانگین نمره خریداران

\item
لیست نظرات

\begin{itemize}[label = $\blacksquare$]
\item
نکته: به شکل پیش‌فرض هر فروشنده که کالایی اضافه می‌کند، به صورت یک کالای مجزا در نظر گرفته می‌شود؛ ولی اگر سامانه پیاده‌سازی شده از قابلیت چند فروشنده برای یک محصول پشتیبانی کند نمرهٔ امتیازی دارد.
\end{itemize}

\end{itemize}
\newpage
\item
نظر:

\begin{itemize}
\item
کاربر نظر دهنده

\item
کالا

\item
متن نظر

\item
وضعیت نظر (در انتظار تایید/تایید شده/تایید نشده توسط مدیر)

\item
آیا نظر دهنده محصول را خریده است.

\end{itemize}

\item
نمره:

\begin{itemize}

\item
کاربر (تنها کاربری که محصول را خریده)

\item
امتیاز

\item
کالا

\end{itemize}


\end{itemize}

\newpage

\subsubsection*{{\titr نمای کلی رابط کاربری:}}
\addcontentsline{toc}{subsubsection}{{\fehrestContent نمای کلی رابط کاربری}}

\begin{itemize}

\item
صفحهٔ اصلی:

\begin{itemize}

\item
ناحیهٔ کاربری

\item
صفحه محصولات

\item
حراج‌ها

\end{itemize}

\item
ناحیهٔ‌ کاربری:

این بخش مشتمل بر دو حالت است:

\begin{enumerate}

\item
لاگین کرده: در این حالت مشخصات و دسترسی‌های حساب کاربری نشان داده می‌شود.

\item
لاگین نکرده: امکان انتقال به صفحهٔ ثبت‌نام یا لاگین را برای کاربر مهیا می‌کند.
\end{enumerate}


\item
پنل ثبت‌نام:

مشخصات فردی و مشخصات مربوط به نقش که در بخش حساب کاربری مطرح شده گرفته می‌شود.

\begin{itemize}[label={$\blacksquare$}]
\item
نکته ۱: ثبت‌نام فقط برای نقش‌های خریدار و فروشنده قابل انجام است.

\item
نکته ۲: ساخت اکانت مدیر تنها توسط مدیر قابل انجام است.

\item
نکته ۳: یک فروشنده هنگام ثبت‌نام نقش فروشنده را درخواست می‌کند و پس از تایید مدیر دسترسی‌های فروشنده را خواهد داشت.

\end{itemize}

\item
صفحه محصولات:

\begin{itemize}
\item
دسته‌‌ها \lr{(category)}

\begin{itemize}[label={$\blacksquare$}]
\item
لیست کالاها

\item
جستجو

\end{itemize}
\end{itemize}

\newpage
\item
حراج‌ها:

\begin{itemize}
\item
لیست کالاهای حراج‌شده.

\item
جستجو
\end{itemize}

\item
جستجو:

\begin{itemize}
\item
فیلتر بر اساس ویژگی‌های عمومی

\item
فیلترهای مختص هر دسته (روی ویژگی‌های مختلف دسته)

\begin{itemize}[label={$\blacksquare$}]
\item
می‌تواند بصورت انتخابی (مانند موجود بودن کالا، یا برند کالا) و یا بصورت بازه‌ای (مانند بازه‌قیمت) باشد.
\end{itemize}
\end{itemize}

\end{itemize}

\newpage

\subsection*{{\titr نقش‌ها}}
\addcontentsline{toc}{subsection}{{\fehrestContent نقش‌ها}}

در این فاز، ۳ نقش مد نظر است.

\begin{itemize}
\item
خریدار:

\begin{itemize}

\item
داده‌های مورد نیاز:

\begin{itemize}[label={$\blacksquare$}]

\item
اطلاعات شخصی حساب کاربری

\item
سبد خرید

\item
سابقه خرید

\item
اعتبار حساب کاربری

\end{itemize}

\item
قابلیت‌ها:

\begin{itemize}[label={$\blacksquare$}]

\item
تغییر اطلاعات شخصی

\item
مشاهدهٔ محصولات

\item
فیلتر کردن و جستجو در میان محصولات

\item
مقایسهٔ دو محصول
\item
خرید

\end{itemize}

\end{itemize}


\item
فروشنده:

\begin{itemize}
\item
اطلاعات شخصی حساب کاربری و اطلاعات مربوط به شرکت

\item
لیست سابقه فروش

\item
تغییر اطلاعات شخصی

\item
لیست محصولات فروشی

\begin{itemize}[label={$\blacksquare$}]
\item
ویرایش هر محصول (شامل ویرایش هر چیزی) که در نهایت به صورت یک درخواست به مدیران ارسال می‌شود.


\item
درخواست حذف یک محصول

\end{itemize}

\item
درخواست افزودن محصول 
\newpage

\item
لیست حراج‌ها

\begin{itemize}[label={$\blacksquare$}]
\item
ویرایش حراج‌ها که بصورت یک درخواست به مدیران ارسال می‌شود.
\end{itemize}


\item
درخواست افزودن یک حراج

\end{itemize}


\item
مدیر:

\begin{itemize}


\item
اطلاعات شخصی حساب کاربری

\item
تغییر اطلاعات شخصی

\item
مشاهده لیست درخواست‌ها

\begin{itemize}[label={$\blacksquare$}]
\item
درخواست‌های ثبت‌نام اکانت فروشنده

\item
درخواست افزودن/ویرایش محصول

\item
درخواست افزودن/ویرایش حراج

\end{itemize}

\item
لیست کد تخفیف‌ها

\begin{itemize}[label={$\blacksquare$}]
\item
ویرایش هر کد تخفیف
\end{itemize}

\item
ایجاد کد تخفیف

\item
مشاهده لیست تمامی کاربران

\begin{itemize}[label={$\blacksquare$}]
\item
امکان حذف یک کاربر
\end{itemize}

\item
امکان افزودن اکانت مدیر

\item
لیست دسته‌ها

\begin{itemize}[label={$\blacksquare$}]
\item
ویرایش یک دسته
\end{itemize}

\item
افزودن دسته
\end{itemize}


\end{itemize}

\newpage

\subsection*{{\titr فرآیندها}}
\addcontentsline{toc}{subsection}{{\fehrestContent فرآیندها}}

\begin{itemize}

\item
ثبت‌نام:

\begin{itemize}
\item
وارد کردن اطلاعات شخصی

\item
وارد کردن اطلاعات مربوط به نقش درخواست شده

\end{itemize}

\item
ورود:
\begin{itemize}
\item
وارد کردن نام کاربری و گذرواژه

\end{itemize}

\item
ویرایش اطلاعات کاربری:

\begin{itemize}
\item
از طریق ورود به پنل کاربری، کاربر می‌تواند مشخصات خود (بجز نام کاربری) را تغییر دهد.

\end{itemize}

\item
مشاهدهٔ محصولات، فیلتر کردن و جستجو (بدون ورود به حساب نیز قابل دسترسی است):

روند کار به این صورت است:

\begin{enumerate}

\item
کاربر صفحه محصولات می‌شود.

\item
دستهٔ مورد نظر را انتخاب می‌کند.

\item
فیلتر‌های مورد نظر را وارد می‌کند.

\item
در صورت نیاز جستجو می‌کند یا صفحات بعدی لیست محصولات را بررسی می‌کند. 

\item
امکان رفع فیلترهای اعمال شده نیز می‌بایست وجود داشته باشد.

\end{enumerate}

\item
مقایسه:

\begin{itemize}

\item
کاربر مجاز است دو محصول از یک دسته‌بندی را با یکدیگر مقایسه کند.

\item
پس از اعمال مقایسه، تمامی ویژگی‌های دو محصول در کنار هم می‌آیند. در صورتی که اعداد یا توضیحات یکی از مشخصه‌های یکی از محصولات ناقص بود، خالی نمایش داده می‌شود.

\end{itemize}

\item
افزودن به سبد خرید:

\begin{itemize}

\item
اگر کاربر تا الان وارد سامانه نشده است، وارد صفحهٔ ثبت‌نام/ورود می‌شود.

\begin{itemize}[label = {$\bigstar$}]

\item
قابلیت امتیازی: بدون وارد شدن به حساب بتوان سبد خرید داشت و تنها هنگام پرداخت، ورود به حساب انجام شود.


\end{itemize}

\item
در صورت تایید، محصول وارد سبد خرید می‌شود.
\end{itemize}

\item
سبد خرید

\begin{itemize}

\item
مشاهده سبد خرید

\item
در صورت تایید، وارد صفحهٔ مشخصات دریافت‌کننده \lr{(shipping)} می‌شود.

\item
در صورت تایید، وارد صفحهٔ ثبت کد تخفیف و تایید نهایی خرید می‌شود. (استفاده از کد تخفیف برای هر فاکتور تنها یکبار قابل استفاده است)

\item
تایید کسر مبلغ از کیف پول (یا نبود موجودی کافی)

\item
بازگشت به صفحهٔ اصلی

\end{itemize}

\item
فروش:

\begin{itemize}

\item
از طریق پنل کاربری کاربر فروشنده قابل انجام است.

\item
هر فروشنده محصول خود را به عنوان کالای جدید معرفی می‌کند.

\begin{itemize}[label={$\blacksquare$}]
\item
نکته:‌ در صورتی که ویژگی چند فروشنده برای یک محصول پشتیبانی می‌شود، فروشنده درخواست خود برای فروشنده شدن یک کالای موجود ثبت می‌کند.
\end{itemize}

\item
اعلام کردن قیمت، تعداد موجودی، توضیحات و ویژگی‌های آن کالا

\end{itemize}
\item
ویرایش اطلاعات یک محصول:

\begin{itemize}

\item
از طریق پنل کاربری کاربر فروشنده/مدیر قابل انجام است.

\item
درخواست ویرایش اطلاعات کالای خود را پس از ثبت مشخصات جدید ثبت می‌کند.
\end{itemize}

\newpage

\item
ایجاد دسته‌بندی جدید:

\begin{itemize}

\item
این فرایند تنها از طریق پنل کاربری مدیر قابل انجام است.

\item
نام و ویژگی‌های آن دسته را تعیین می‌کند.

\end{itemize}
\item
ایجاد تخفیف کد دار:

\begin{itemize}

\item
این فرایند تنها از طریق پنل کاربری مدیر قابل انجام است.

\item
تمام مشخصات تخفیف را تعیین می‌کند.

\item[{$\bigstar$}]
قابلیت امتیازی: بر اساس رویداد خاصی (مثلا مجموع خرید بیشتر از ۱ میلیون تومان شود) به شخص کد تخفیف داده‌ شود.

\item[{$\bigstar$}]
قابلیت امتیازی: به صورت دوره‌ای به تعدادی کاربر اتفاقی کد تخفیف داده‌ شود.

\end{itemize}

\item
ویرایش تخفیف کد دار:

\begin{itemize}

\item
این فرایند تنها از طریق پنل کاربری مدیر قابل انجام است.

\item
مشخصات تخفیف (بجز کد) را تغییر می‌دهد.

\end{itemize}

\item
ایجاد حراج:

\begin{itemize}
\item
این فرایند توسط فروشنده قابل انجام است.

\item
لیست محصولات مورد تخفیف، درصد و سقف مقدار تخفیف و ددلاین را مشخص می‌کند.

\item
درخواست را به مدیر ارسال می‌کند.


\end{itemize}
\item
ویرایش حراج:


\begin{itemize}

\item
این فرایند توسط فروشنده قابل انجام است.

\item
می‌تواند اطلاعات مربوط به یک حراج خود را تغییر دهد.

\item
درخواست را به مدیر ارسال می‌کند.

\end{itemize}

\newpage

\item
نمره دهی:


\begin{itemize}

\item
کاربر وارد ناحیه کاربری می‌شود.

\item
سابقهٔ خرید خود را مشاهده می‌کند و به محصول مورد نظر نمره می‌دهد.

\begin{itemize}[label = {$\blacksquare$}]

\item
نکته: واضح است که محصول باید توسط کاربر خریده شده باشد تا بتواند به آن امتیاز دهد.

\end{itemize}



\end{itemize}
\item
ثبت نظر:

\begin{itemize}

\item
وارد صفحهٔ محصول مورد نظر می‌شود.

\item
نظر خود را درمورد آن محصول می‌نویسد.

\item
سامانه به صورت خودکار تعیین می‌کند که نظر دهنده کیست و آیا این محصول را خریده است یا خیر.

\end{itemize}

\end{itemize}


\newpage

\section*{{\titr دستورات مورد نیاز}}
\addcontentsline{toc}{section}{{\fehrestContent دستورات مورد نیاز}}

\subsection*{{\titr نکات مهم:}}
\addcontentsline{toc}{subsection}{{\fehrestContent نکات مهم:}}

\begin{itemize}[label={$\bigstar$}]

\item
در این بخش، دستورات مورد نیاز در هر صفحه آمده‌ است. توجه کنید که دستورات به صورت کلی توضیح داده‌شده‌اند و هر جا به جزئیات اشاره نشده‌ است، می‌توانید دلخواه عمل کنید.

\item
در همه منوها باید امکان لاگین و رجیستر برای شخص لاگین‌نشده و امکان لاگ اوت برای شخص لاگین شده وجود داشته باشد و پس از اتمام فرایند به صفحه‌ای که قبلا بوده برگردد.

\item
در همه منوها یک دستور \lr{help} موجود باشد که لیست دستورات آن منو را نشان دهد.

\item
در همه منوها هر جا لیستی نمایش داده می‌شود، بتوان از طریق دستور \lr{sort by [field]} نمایش لیست را تغییر داد.

\item
در هر یک از زیر منوهای برنامه با دستور \lr{back} به منوی قبلی برمی‌گردد.



\end{itemize}

\newpage


\subsection*{{\titr صفحه لاگین/رجیستر}}
\addcontentsline{toc}{subsection}{{\fehrestContent صفحه لاگین/رجیستر:}}

هرگاه کاربری که مهمان باشد یا از نوع خریدار نباشد، سعی کند وارد ناحیه کاربری شود یا عملیات افزودن به سبد خرید را انجام دهد، یا در حالتی که سبد خرید بدون لاگین در دسترس باشد، بخواهد خرید را انجام دهد، باید با نمایش پیغام خطای مناسب به این صفحه منتقل شده و پس از انجام لاگین/رجیستر، به صفحه‌ای که قبلا بوده برگردد.

دستورات این منو بدین شرح هستند:


\begin{mybox}[colback=yellow]{دستور}


\begin{latin}

create account [type] [username]

\end{latin}

\end{mybox}

منظور از \lr{type}، نوع حساب کاربری است و به این صورت است:

\begin{enumerate}

\item
 خریدار
 
 \item
 فروشنده
 
 \item
 مدیر - البته این فقط یکبار امکان پذیر است و سایر مدیران را مدیر اصلی اضافه می‌کند.
 
 \end{enumerate}
 
پس از زدن دستور، از کاربر \lr{password} خواسته می‌شود. سپس اطلاعات شخصی مربوط به نوع حساب کاربری مورد نظر خواسته‌ می‌شود.

(در صورتی که یوزرنیم از قبل وجود داشت، پیغام خطای مناسبی نمایش داده شود.)
هر حساب یک \lr{username} دارد که یکتاست.

\hrulefill

\begin{mybox}[colback=yellow]{دستور}


\begin{latin}

login [username]

\end{latin}

\end{mybox}

پس از زدن دستور، از کاربر \lr{password} خواسته می‌شود.

در صورت درستی پسورد به صفحه‌ای که قبلا بوده یا می‌خواسته برود، می‌رود.

(در صورت عدم وجود حسابی با این \lr{username} یا غلط بودن \lr{password}،  خطای مناسبی نمایش داده شود.)

\newpage

\subsection*{{\titr صفحه اصلی > ناحیه کاربری}}
\addcontentsline{toc}{subsection}{{\fehrestContent صفحه اصلی > ناحیه کاربری }}

\subsubsection*{{\titr «حساب مدیر»}}
\addcontentsline{toc}{subsubsection}{{\fehrestContent «حساب مدیر»}}


دستورات پنل مدیریت به صورت زیر است:

\begin{mybox}[colback=yellow]{دستور}

\begin{latin}

view personal info

\end{latin}

\end{mybox}

اطلاعات شخصی کاربر را نمایش می‌دهد. حال کاربر می‌تواند اطلاعات را ویرایش کند یا به منوی اصلی برگردد.

\begin{mybox}[colback=yellow]{دستور}


\begin{latin}

\begin{itemize}[label = {$\Rightarrow$}]

\item
edit [field] 

\end{itemize}

\end{latin}

\end{mybox}




برای ویرایش اطلاعات. \lr{field} ها می‌توانند هر اطلاعات شخصی‌ای بجز \lr{username} باشند.

\hrulefill


\begin{mybox}[colback=yellow]{دستور}

\begin{latin}

manage users

\end{latin}

\end{mybox}


با این دستور لیست کاربران را نمایش می‌دهد. مدیر می تواند از طریق دستورات زیر اکانت کاربر را حذف کند یا نقش کاربر را تغییر دهد.

\begin{mybox}[colback=brilliantlavender]{دستور}


\begin{latin}

\begin{itemize}[label = {$\Rightarrow$}]

\item
view [username]

\end{itemize}

\end{latin}

\end{mybox}




\begin{mybox}[colback=brilliantlavender]{دستور}


\begin{latin}

\begin{itemize}[label = {$\Rightarrow$}]

\item
delete user [username]

\end{itemize}

\end{latin}

\end{mybox}


\begin{mybox}[colback=brilliantlavender]{دستور}


\begin{latin}

\begin{itemize}[label = {$\Rightarrow$}]

\item
create manager profile

\end{itemize}

\end{latin}

\end{mybox}

با این دستور، مشخصات یک مدیر جدید از کاربر گرفته می‌شود و مدیر ثبت نام می‌شود.
(در صورتی وجود خطا، پیغام خطای مناسبی نمایش داده شود.)

\hrulefill


\begin{mybox}[colback=yellow]{دستور}

\begin{latin}

manage all products

\end{latin}

\end{mybox}


امکان حذف محصول را با دستور زیر فراهم می‌کند:


\begin{mybox}[colback=brilliantlavender]{دستور}


\begin{latin}

\begin{itemize}[label = {$\Rightarrow$}]

\item
remove [productId]

\end{itemize}

\end{latin}

\end{mybox}

(در صورتی که \lr{productId} وجود نداشت، پیغام خطای مناسبی نمایش داده شود.)


\hrulefill

\begin{mybox}[colback=yellow]{دستور}

\begin{latin}

create discount code

\end{latin}

\end{mybox}

پس از این دستور، اطلاعات مورد نیاز تخفیف کددار گرفته می‌شود و در صورت اعتبار فیلدهای ورودی کد تخفیف ساخته می‌شود.


\hrulefill


\begin{mybox}[colback=yellow]{دستور}

\begin{latin}

view discount codes

\end{latin}

\end{mybox}

لیست تمامی تخفیف‌های فروشگاه را نمایش می‌دهد.


\begin{mybox}[colback=brilliantlavender]{دستور}


\begin{latin}

\begin{itemize}[label = {$\Rightarrow$}]

\item
view discount code [code]

\end{itemize}

\end{latin}

\end{mybox}

مشخصات کد مورد نظر نمایش داده می‌شود.


\begin{mybox}[colback=brilliantlavender]{دستور}


\begin{latin}

\begin{itemize}[label = {$\Rightarrow$}]

\item
edit discount code [code]

\end{itemize}

\end{latin}

\end{mybox}


مشخصات کد مورد نظر را می‌توان ویرایش کرد.

\begin{mybox}[colback=brilliantlavender]{دستور}


\begin{latin}

\begin{itemize}[label = {$\Rightarrow$}]

\item
remove discount code [code]

\end{itemize}

\end{latin}

\end{mybox}


کد تخفیف حذف می‌شود.


\hrulefill


\begin{mybox}[colback=yellow]{دستور}

\begin{latin}

manage requests

\end{latin}

\end{mybox}

لیست درخواست‌های ارسال شده به مدیر را نمایش می‌دهد.



\begin{mybox}[colback=brilliantlavender]{دستور}

\begin{latin}

\begin{itemize}[label = {$\Rightarrow$}]

\item
details [requestId]

\end{itemize}

\end{latin}

\end{mybox}

با این دستور دستور زیر تمام مشخصات درخواست مورد نظر نمایش داده می‌شود.


\begin{mybox}[colback=brilliantlavender]{دستور}


\begin{latin}

\begin{itemize}[label = {$\Rightarrow$}]

\item
accept [requestId]

\item
decline [requestId]

\end{itemize}

\end{latin}

\end{mybox}

مدیر با این دستورات درخواست را تایید یا رد می کند.

\hrulefill

\newpage

\begin{mybox}[colback=yellow]{دستور}

\begin{latin}

manage categories

\end{latin}

\end{mybox}

دسته بندی‌های موجود کالاها را نمایش می‌دهد (برای ویرایش/اضافه کردن دسته بندی‌ها)


\begin{mybox}[colback=brilliantlavender]{دستور}

\begin{latin}

\begin{itemize}[label = {$\Rightarrow$}]

\item
edit [category]

\end{itemize}

\end{latin}

\end{mybox}

اطلاعاتی نظیر نام دسته، و ویژگی‌های دسته را می‌تواند تغییر دهد.


\begin{mybox}[colback=brilliantlavender]{دستور}

\begin{latin}

\begin{itemize}[label = {$\Rightarrow$}]

\item
add [category]

\end{itemize}

\end{latin}

\end{mybox}

سپس اطلاعات مربوط به دسته شامل ویژگی ها خواسته می‌شود.

\begin{mybox}[colback=brilliantlavender]{دستور}

\begin{latin}

\begin{itemize}[label = {$\Rightarrow$}]

\item
remove [category]

\end{itemize}

\end{latin}

\end{mybox}

یک دسته و تمام محصولات آن را حذف می‌کند.

\newpage


\subsubsection*{{\titr «حساب فروشنده»}}
\addcontentsline{toc}{subsubsection}{{\fehrestContent «حساب فروشنده»}}

دستورات مربوط به ناحیه کاربری فروشنده به صورت زیر است:

\begin{mybox}[colback=yellow]{دستور}

\begin{latin}

view personal info

\end{latin}

\end{mybox}

اطلاعات شخصی کاربر را نمایش می‌دهد. حال کاربر می تواند اطلاعات را ویرایش کند یا به منوی اصلی برگردد.

\begin{mybox}[colback=brilliantlavender]{دستور}

\begin{latin}

\begin{itemize}[label = {$\Rightarrow$}]

\item
edit [field] 

\end{itemize}

\end{latin}

\end{mybox}

برای ویرایش اطلاعات. field ها می‌توانند هر اطلاعات شخصی‌ای بجز username باشند.

\hrulefill

\begin{mybox}[colback=yellow]{دستور}

\begin{latin}

view company information

\end{latin}

\end{mybox}

مشخصات شرکت، کارگاه و یا کارخانهٔ فروشنده را نمایش می‌دهد. (اسم ، در صورت وجود سایر مشخصات) 

\hrulefill

\begin{mybox}[colback=yellow]{دستور}

\begin{latin}

view sales history

\end{latin}

\end{mybox}

سابقه فروش محصولات فروشنده را نمایش می‌دهد.


\hrulefill

\newpage

\begin{mybox}[colback=yellow]{دستور}

\begin{latin}

manage products

\end{latin}

\end{mybox}

لیست محصولات فروشنده را نمایش می‌دهد و امکان مشاهده جزئیات خریداران محصول و ویرایش مشخصات هر محصول و قرار دادن تخفیف را با دستورات زیر فراهم می‌کند:


\begin{mybox}[colback=brilliantlavender]{دستور}

\begin{latin}

\begin{itemize}[label = {$\Rightarrow$}]

\item
view [productId]

\end{itemize}

\end{latin}

\end{mybox}

\begin{mybox}[colback=brilliantlavender]{دستور}

\begin{latin}

\begin{itemize}[label = {$\Rightarrow$}]

\item
view buyers [productId]

\end{itemize}

\end{latin}

\end{mybox}

\begin{mybox}[colback=brilliantlavender]{دستور}

\begin{latin}

\begin{itemize}[label = {$\Rightarrow$}]

\item
edit [productId]

\end{itemize}

\end{latin}

\end{mybox}

 پس از این دستور، فیلد‌های مورد ویرایش و مقادیر جدید آن‌ها گرفته می‌شود و پس از تایید، درخواست تغییر برای مدیر ارسال می‌شود تا در صورت موافقت تغییر اعمال شود.

\hrulefill

\begin{mybox}[colback=yellow]{دستور}

\begin{latin}

add product 

\end{latin}

\end{mybox}


پس از این دستور مشخصات یک محصول جدید از کاربر گرفته می‌شود و یک درخواست برای مدیر ارسال می‌شود تا در صورت موافقت، آن کالا به لیست اضافه شود.

\hrulefill

\begin{mybox}[colback=yellow]{دستور}

\begin{latin}

remove product [productId] 

\end{latin}

\end{mybox}

محصول مورد نظر را حذف می‌کند.


\hrulefill

\newpage

\begin{mybox}[colback=yellow]{دستور}

\begin{latin}

show categories 

\end{latin}

\end{mybox}

پس از این دستور تمامی دسته‌های موجود را به کاربر نمایش می‌دهد.

\hrulefill

\begin{mybox}[colback=yellow]{دستور}

\begin{latin}

view offs

\end{latin}

\end{mybox}

لیست حراج‌های فروشنده را نمایش می‌دهد.


\begin{mybox}[colback=brilliantlavender]{دستور}

\begin{latin}

\begin{itemize}[label = {$\Rightarrow$}]

\item
view [offId]

\end{itemize}

\end{latin}

\end{mybox}

مشخصات off مورد نظر را نمایش می‌دهد.



\begin{mybox}[colback=brilliantlavender]{دستور}

\begin{latin}

\begin{itemize}[label = {$\Rightarrow$}]

\item
edit [offId]

\end{itemize}

\end{latin}

\end{mybox}

مشخصات off مورد نظر را می‌تواند ویرایش کند.


\begin{mybox}[colback=brilliantlavender]{دستور}

\begin{latin}

\begin{itemize}[label = {$\Rightarrow$}]

\item
add off

\end{itemize}

\end{latin}

\end{mybox}

برای اضافه کردن off جدید؛ پس از این دستور مشخصات حراج دریافت می‌شود.

\begin{itemize}[label={$\blacksquare$}]
\item
نکته: دستورات edit و add مستقیماً اعمال نمی‌شوند و بصورت درخواست به مدیر ارسال می‌شوند.

\end{itemize}


\hrulefill

\begin{mybox}[colback=yellow]{دستور}

\begin{latin}

view balance

\end{latin}

\end{mybox}

میزان اعتبار فروشنده را نشان می‌دهد.


\newpage

\subsubsection*{{\titr «حساب خریدار»}}
\addcontentsline{toc}{subsubsection}{{\fehrestContent «حساب خریدار»}}

دستورات مربوط به ناحیه کاربری خریدار به صورت زیر است:


\begin{mybox}[colback=yellow]{دستور}

\begin{latin}

view personal info

\end{latin}

\end{mybox}

اطلاعات شخصی کاربر را نمایش می‌دهد. حال کاربر می‌تواند اطلاعات را ویرایش کند یا به منوی اصلی برگردد:

\begin{mybox}[colback=brilliantlavender]{دستور}

\begin{latin}

\begin{itemize}[label = {$\Rightarrow$}]

\item
edit [field] 

\end{itemize}

\end{latin}

\end{mybox}

برای ویرایش اطلاعات. field ها می‌توانند هر اطلاعات شخصی‌ای به‌جز username باشند.

\hrulefill

\begin{mybox}[colback=yellow]{دستور}

\begin{latin}

view cart

\end{latin}

\end{mybox}

سبد خرید را نمایش می‌دهد و کاربر می‌تواند محصولات را مشاهده یا از سبد حذف کند و یا خرید خود را نهایی کند.


\begin{mybox}[colback=brilliantlavender]{دستور}

\begin{latin}

\begin{itemize}[label = {$\Rightarrow$}]

\item
show products

\end{itemize}

\end{latin}

\end{mybox}

محصولات موجود در سبد خرید را نشان می‌دهد.


\begin{mybox}[colback=brilliantlavender]{دستور}

\begin{latin}

\begin{itemize}[label = {$\Rightarrow$}]

\item
view [productId]

\end{itemize}

\end{latin}

\end{mybox}

ورود به صفحهٔ محصول


\begin{mybox}[colback=brilliantlavender]{دستور}

\begin{latin}

\begin{itemize}[label = {$\Rightarrow$}]

\item
increase [productId]

\end{itemize}

\end{latin}

\end{mybox}

افزایش تعداد یک محصول



\begin{mybox}[colback=brilliantlavender]{دستور}

\begin{latin}

\begin{itemize}[label = {$\Rightarrow$}]

\item
decrease [productId]

\end{itemize}

\end{latin}

\end{mybox}

کاهش تعداد یک محصول (اگر تعداد آن صفر شد،‌ آن محصول از سبد خرید حذف می‌شود.)




\begin{mybox}[colback=brilliantlavender]{دستور}

\begin{latin}

\begin{itemize}[label = {$\Rightarrow$}]

\item
show total price

\end{itemize}

\end{latin}

\end{mybox}

نشان دادن مجموع قیمت محصولات


\begin{mybox}[colback=brilliantlavender]{دستور}

\begin{latin}

\begin{itemize}[label = {$\Rightarrow$}]

\item
purchase

\end{itemize}

\end{latin}

\end{mybox}

ورود به صفحهٔ پرداخت

\hrulefill

\begin{mybox}[colback=yellow]{دستور}

\begin{latin}

purchase

\end{latin}

\end{mybox}

پس از وارد کردن این کلمه در صفحهٔ سبد خرید به ترتیب باید وارد صفحات زیر شود:


\begin{latin}

\begin{enumerate}

\item
receiver information

\item
discount code

\item
payment


\end{enumerate}


\end{latin}

که به ترتیب بدین شرح هستند:

\begin{enumerate}

\item
صفحه وارد کردن اطلاعات خریدار اعم از آدرس، تلفن و …

\item
صفحهٔ ثبت کد تخفیف در صورت داشتن کد و تایید اعتبار کد

\item
صفحه پرداخت وجه و گزارش نهایی از تراکنش‌ (اگر موفق بود تایید کسر مبلغ در غیر این صورت خطای مناسب نمایش داده‌ شود.)


\end{enumerate}

\hrulefill


\begin{mybox}[colback=yellow]{دستور}

\begin{latin}

view orders

\end{latin}

\end{mybox}

سابقۀ خرید کاربر را نمایش می‌دهد.

امکان مشاهده جزئیات یا نظردهی محصولات خریداری شده، از طریق دو دستور زیر وجود دارد:

\begin{mybox}[colback=brilliantlavender]{دستور}

\begin{latin}

\begin{itemize}[label = {$\Rightarrow$}]

\item
show order [orderId]

\item
rate [productId] [1-5]

\end{itemize}

\end{latin}

\end{mybox}

اگر چنین محصولی خریداری نشده بود، پیغام خطای مناسب چاپ شود.

\hrulefill


\begin{mybox}[colback=yellow]{دستور}

\begin{latin}

view balance

\end{latin}

\end{mybox}

میزان اعتبار کاربر را نشان می‌دهد.

\hrulefill


\begin{mybox}[colback=yellow]{دستور}

\begin{latin}

view discount codes

\end{latin}

\end{mybox}

کد تخفیف‌های کاربر را نشان می‌دهد.



\newpage

\subsection*{{\titr صفحهٔ محصولات}}
\addcontentsline{toc}{subsection}{{\fehrestContent صفحهٔ محصولات}}

\begin{mybox}[colback=yellow]{دستور}

\begin{latin}

products

\end{latin}

\end{mybox}


با وارد کردن این دستور در صفحهٔ اصلی هر ۳ نوع از کاربرها یا حتی قبل از لاگین، وارد صفحهٔ محصولات می‌شود.

\hrulefill

\begin{mybox}[colback=yellow]{دستور}

\begin{latin}

view categories

\end{latin}

\end{mybox}

لیست دسته‌های محصولات را نمایش می‌دهد.

\hrulefill


\begin{mybox}[colback=yellow]{دستور}

\begin{latin}

filtering

\end{latin}

\end{mybox}

این دستورات مربوط به فیلتر کردن هستند:


\begin{mybox}[colback=brilliantlavender]{دستور}

\begin{latin}

\begin{itemize}[label = {$\Rightarrow$}]

\item
show available filters

\end{itemize}

\end{latin}

\end{mybox}

مواردی که می‌توان بر حسب آن فیلتر کرد. ( فیلتر بر حسب کتگوری، نام محصول و … )



\begin{mybox}[colback=brilliantlavender]{دستور}

\begin{latin}

\begin{itemize}[label = {$\Rightarrow$}]

\item
filter [an available filter]

\end{itemize}

\end{latin}

\end{mybox}

محصولات را بر حسب نوع فیلتر ورودی ، فیلتر کرده و نمایش می‌دهد.



\begin{mybox}[colback=brilliantlavender]{دستور}

\begin{latin}

\begin{itemize}[label = {$\Rightarrow$}]

\item
current filters

\end{itemize}

\end{latin}

\end{mybox}

فیلترهای انتخابی را نشان می‌دهد.


\begin{mybox}[colback=brilliantlavender]{دستور}

\begin{latin}

\begin{itemize}[label = {$\Rightarrow$}]

\item
disable filter [a selected filter]

\end{itemize}

\end{latin}

\end{mybox}

فیلتر مورد نظر را غیر فعال می‌کند. (به جز دسته که همه‌شان را نمی‌شود حذف کرد)

\hrulefill

\begin{mybox}[colback=yellow]{دستور}

\begin{latin}

sorting

\end{latin}

\end{mybox}

وارد قسمت مرتب‌سازی محصولات می‌شود.



\begin{mybox}[colback=brilliantlavender]{دستور}

\begin{latin}

\begin{itemize}[label = {$\Rightarrow$}]

\item
show available sorts

\end{itemize}

\end{latin}

\end{mybox}

مواردی که می‌توان بر حسب آن محصولات را مرتب شده نمایش داد. (زمان، نمره، تعداد بازدید)



\begin{mybox}[colback=brilliantlavender]{دستور}

\begin{latin}

\begin{itemize}[label = {$\Rightarrow$}]

\item
sort [an available sort]

\end{itemize}

\end{latin}

\end{mybox}

مرتب سازی محصولات بر حسب المان وارد شده.




\begin{mybox}[colback=brilliantlavender]{دستور}

\begin{latin}

\begin{itemize}[label = {$\Rightarrow$}]

\item
current sort

\end{itemize}

\end{latin}

\end{mybox}


سورت‌ انتخابی را نشان می‌دهد.


\begin{mybox}[colback=brilliantlavender]{دستور}

\begin{latin}

\begin{itemize}[label = {$\Rightarrow$}]

\item
disable sort

\end{itemize}

\end{latin}

\end{mybox}

سورت مورد نظر را غیر فعال می‌کند. (پیش‌فرض روی تعداد بازدید)

\hrulefill

\newpage

\begin{mybox}[colback=yellow]{دستور}

\begin{latin}

show products

\end{latin}

\end{mybox}

محصولات موجود در فروشگاه را نمایش می‌دهد. (در صورتی که فیلتری انتخاب شده باشد، فیلتر شدهٔ محصولات را بر اساس نوع مرتب‌سازی انتخاب شده توسط کاربر مرتب کرده و نشان می‌دهد.‌)

\hrulefill




\begin{mybox}[colback=yellow]{دستور}

\begin{latin}

show product [productId]

\end{latin}

\end{mybox}

وارد صفحهٔ محصول مورد نظر می‌شود.

\newpage
\subsection*{{\titr صفحهٔ محصول}}
\addcontentsline{toc}{subsection}{{\fehrestContent صفحهٔ محصول}}



\begin{mybox}[colback=yellow]{دستور}

\begin{latin}

digest

\end{latin}

\end{mybox}

اطلاعات محصول را به صورت خلاصه نمایش می‌دهد که شامل توضیحات و قیمت و تخفیف و کتگوری و فروشنده و میانگین نمرات می‌شود.


\begin{mybox}[colback=brilliantlavender]{دستور}

\begin{latin}

\begin{itemize}[label = {$\Rightarrow$}]

\item
add to cart

\end{itemize}

\end{latin}

\end{mybox}

محصول را به سبد خرید اضافه می‌کند. (در صورت login نبودن به صفحه ورود برود)


\begin{mybox}[colback=brilliantlavender]{دستور}

\begin{latin}

\begin{itemize}[label = {$\Rightarrow$}]

\item
select seller [seller\_username]

\end{itemize}

\end{latin}

\end{mybox}

مشخص می‌کنیم از چه فروشنده‌ای بخریم. (در صورتی که قابلیت امتیازی مربوطه پیاده‌سازی شده‌ باشد.)

\hrulefill

\begin{mybox}[colback=yellow]{دستور}

\begin{latin}

attributes

\end{latin}

\end{mybox}

تمام ویژگی‌های محصول هم عمومی و هم بر اساس کتگوری را با ذکر مقدار نمایش می‌دهد.


\hrulefill

\begin{mybox}[colback=yellow]{دستور}

\begin{latin}

compare  [productID]

\end{latin}

\end{mybox}

ویژگی‌های دو محصول را کنار هم نمایش می‌دهد.


\hrulefill

\begin{mybox}[colback=yellow]{دستور}

\begin{latin}

Comments

\end{latin}

\end{mybox}

نظرات و امتیاز محصول را نشان می‌کند.

\begin{mybox}[colback=brilliantlavender]{دستور}

\begin{latin}

\begin{itemize}[label = {$\Rightarrow$}]

\item
Add comment

Title:

Content:

\end{itemize}

\end{latin}

\end{mybox}

با همین مشخصات یک نظر به محصول اضافه می‌کند.

\newpage

\subsection*{{\titr صفحهٔ حراج‌ها}}
\addcontentsline{toc}{subsection}{{\fehrestContent صفحهٔ حراج‌ها}}


\begin{mybox}[colback=yellow]{دستور}

\begin{latin}

offs

\end{latin}

\end{mybox}

با وارد کردن این دستور در صفحهٔ اصلی هر ۳ نوع از کاربرها یا حتی قبل از لاگین، وارد صفحهٔ حراج‌ها می‌شود. در این صفحه لیستی از محصولات دارای حراج با نمایش قیمت قبل و حال حاضر آن‌ها و ذکر جزئیات حراج نمایش داده می‌شود.


\hrulefill

\begin{mybox}[colback=yellow]{دستور}

\begin{latin}

show product [productId]

\end{latin}

\end{mybox}

وارد صفحهٔ محصول مورد نظر می‌شود.


\hrulefill

\begin{itemize}[label={$\blacksquare$}]
\item
نکته: موارد مربوط به فیلتر و جستجو در این صفحه، مشابه صفحهٔ محصولات است.

\end{itemize}

\end{document}







