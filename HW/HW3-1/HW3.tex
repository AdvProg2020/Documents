\documentclass[]{article}
\usepackage{graphicx}
\usepackage[svgnames]{xcolor} 
\usepackage{fancyhdr}

\usepackage{hyperref}
\usepackage{enumitem}
\usepackage[many]{tcolorbox}
\usepackage{listings }
\usepackage[a4paper, total={6in, 8in}]{geometry}
\usepackage{afterpage}
\usepackage{amssymb}
\usepackage{xepersian}
\usepackage[T1]{fontenc}
\usepackage{tikz}
\usepackage[utf8]{inputenc}
\usepackage{PTSerif} 
\usepackage{seqsplit}

\usepackage{listings}
\usepackage{xcolor}
 
\definecolor{codegreen}{rgb}{0,0.6,0}
\definecolor{codegray}{rgb}{0.5,0.5,0.5}
\definecolor{codepurple}{rgb}{0.58,0,0.82}
\definecolor{backcolour}{rgb}{0.95,0.95,0.92}
 
\NewDocumentCommand{\codeword}{v}{
\texttt{\textcolor{blue}{#1}}
}
\lstset{language=C,keywordstyle={\bfseries \color{blue}}}

\lstdefinestyle{mystyle}{
    backgroundcolor=\color{backcolour},   
    commentstyle=\color{codegreen},
    keywordstyle=\color{magenta},
    numberstyle=\tiny\color{codegray},
    stringstyle=\color{codepurple},
    basicstyle=\ttfamily\normalsize,
    breakatwhitespace=false,         
    breaklines=true,                 
    captionpos=b,                    
    keepspaces=true,                 
    numbers=left,                    
    numbersep=5pt,                  
    showspaces=false,                
    showstringspaces=false,
    showtabs=false,                  
    tabsize=2
}

\lstset{style=mystyle}

\settextfont[BoldFont={XB Zar bold.ttf}]{XB Zar.ttf}




\newcommand{\inputsample}[1]{
    ~\\
    \textbf{ورودی نمونه}
    ~\\
    \begin{tcolorbox}[breakable,boxrule=0pt]
        \begin{latin}
            \large{
                #1
            }
        \end{latin}
    \end{tcolorbox}
}

\newcommand{\outputsample}[1]{
    ~\\
    \textbf{خروجی نمونه}

    \begin{tcolorbox}[breakable,boxrule=0pt]
        \begin{latin}
            \large{
                #1
            }
        \end{latin}
    \end{tcolorbox}
}

%%%%%باکس های طراحی شده برای پاسخ نامه ، میتوانید پاسخ را درون باکس قراردهید
\newtcolorbox[auto counter]{solutionbox}{
freelance,
colback=white,
frame code={},
interior titled code={
  \fill[rounded corners=8pt,orange!30]
    (title.south west) --
    (title.south) -- 
    ([yshift=20pt]title.south) --
    ([yshift=20pt,xshift=4cm]title.south) --
    ([xshift=4cm]title.south) --
    (title.south east) {[sharp corners] --
    ([yshift=-6pt]title.south east) -- 
    ([yshift=-6pt]title.south west) } -- cycle;
  \draw[rounded corners=8pt,gray,line width=1pt]
    (title.west|-frame.south west) --
    (title.south west) --
    (title.south) -- 
    ([yshift=20pt]title.south) --
    ([yshift=20pt,xshift=4cm]title.south) --
    ([xshift=4cm]title.south) --
    (title.south east) --
    (title.east|-frame.south east) --
    cycle;
  \node at ([xshift=2cm,yshift=4pt,anchor=south]title.south) 
    {\Large \textbf{پاسخ}};  
  },
title={\mbox{}},
top=12pt,
fontupper=\sffamily\Large,
oversize=0.5cm,
before={\vskip24pt\par\noindent},
after={\par\vskip12pt}
}
\newtcolorbox[auto counter]{solutionboxC}{
freelance,
colback=white,
frame code={},
interior titled code={
  \fill[rounded corners=8pt,orange!30]
    (title.south west) --
    (title.south) -- 
    ([yshift=20pt]title.south) --
    ([yshift=20pt,xshift=4cm]title.south) --
    ([xshift=4cm]title.south) --
    (title.south east) {[sharp corners] --
    ([yshift=-6pt]title.south east) -- 
    ([yshift=-6pt]title.south west) } -- cycle;
  \draw[rounded corners=8pt,gray,line width=1pt]
    (title.west|-frame.south west) --
    (title.south west) --
    (title.south) -- 
    ([yshift=20pt]title.south) --
    ([yshift=20pt,xshift=4cm]title.south) --
    ([xshift=4cm]title.south) --
    (title.south east) --
    (title.east|-frame.south east) --
    cycle;
  \node at ([xshift=2cm,yshift=4pt,anchor=south]title.south) 
    {\Large \textbf{ پاسخ ادامه}};  
  },
title={\mbox{}},
top=12pt,
fontupper=\sffamily\Large,
oversize=0.5cm,
before={\vskip24pt\par\noindent},
after={\par\vskip12pt}
}

\begin{document}


%%% title pages
\begin{titlepage}
\begin{center}
        
\vspace*{0.7cm}

\includegraphics[width=0.4\textwidth]{sharif1.png}\\
\vspace{0.5cm}
\textbf{ \Huge{\emph درس برنامه‌سازی پیشرفته} }\\
\vspace{0.5cm}
\textbf{ \Large{ تمرین سوم بخش اول} }
\vspace{0.2cm}
       
 
      \large \textbf{دانشکده مهندسی کامپیوتر}\\\vspace{0.2cm}
    \large   دانشگاه صنعتی شریف\\\vspace{0.2cm}
       \large   ﻧﯿﻢ سال دوم 99-98 \\\vspace{0.2cm}
      \noindent\rule[1ex]{\linewidth}{1pt}
    مبحث:\\
    \textbf{{گرافیک و جنریک}}

    \vspace{0.20cm}

   مهلت ارسال:\\
    \textbf{{16 خرداد}}\\
    \textbf{{ساعت 23:59}}

    \vspace{0.15cm}
ویراستار فنی:\\
    \textbf{{صابر ظفر‌پور و محمّد فراهانی}}
\end{center}
\end{titlepage}
%%% title pages


%%% header of pages
\newpage
\pagestyle{fancy}
\fancyhf{}
\fancyfoot{}
\cfoot{\thepage}
\chead{گرافیک و جنریک}
\rhead{\includegraphics[width=0.1\textwidth]{sharif.png}}
\lhead{تمرین 1.3 برنامه‌سازی پیشرفته}
%%% header of pages




 \Large \textbf{\\\\
به موارد زیر توجه کنید:}

\begin{itemize}[label=$\ast$]
\item به‌ازای هر سوال در سامانه‌ی کوئرا، یک بخش جداگانه برای بارگذاری برنامه‌ی شما وجود دارد. فایل برنامه‌ی خود با پسوند .zip را در بخش مربوط به هر سوال بارگذاری کنید.
\item پس از ارسال فایل مربوط به هر سوال، سامانه‌ی کوئرا به‌صورت لحظه‌ای برنامه‌ی شما را داوری کرده و نمره‌ی آن سوال را به شما اعلام می‌کند که در صورت کم بودن نمره‌تان، می‌توانید آن را تصحیح کرده و دوباره ارسال کنید. به جز سوال های دوم وسوم که در هنگام تحویل حضوری مورد ارزیابی قراد می گیرد.
\item هم‌فکری و هم‌کاری در پاسخ به تمرینات اشکالی ندارد و حتی توصیه نیز می‌شود؛ ولی پاسخ ارسالی شما باید حتما توسط خود شما نوشته شده‌باشد. در صورت هم‌فکری در مورد یک سوال، نام افراد دیگر را به‌صورت کامنت در ابتدای کد هر سوال بنویسید.  این نکته رو در نظر بگیرید که هم‌فکری تنها مربوط به بخش ایده سوال هست نه پیاده‌سازی آن و در صورت محرز شدن تقلب برای فرد خاطی بدون مسامحه \emph{ منفی نمره تمرین}
منظور می‌گردد. 
\item شما می‌توانید تمامی سوالات و ابهامات خود را در سایت کوئرا در بخش مشخص‌شده برای این تمرین بپرسید.
\item به‌ازای هر روز تاخیر در ارسال پاسخ هر سوال، 30 درصد از نمره‌ی کسب‌شده‌ی شما در آن سوال کم می‌شود. به عنوان مثال اگر پاسخ یک سوال را با دو روز تاخیر ارسال کنید، فقط 40 درصد از نمره‌ای که برای آن سوال گرفته‌اید برای شما لحاظ خواهد شد.
\item در کل شما می‌توانید سه روز تاخیر بدون کسر نمره داشته باشد.
\item مهلت ارسال تمرین تا ساعت 23:59 روز 16 خرداد 1399 است.
\item تنها به یکی از دو سوال دو یا سه پاسخ دهید.
\end{itemize}



\newpage
\section{دردسرهای ناتمام اکبردلاک}
اکبردلاک از دانشجویان پرکار ورودی 98 است که به تازگی برای انتخابات ssc نامزد شده است . ولی به دلیل بسته بودن دانشگاه کل برنامه زندگی اش به هم ریخته است. او حتی یادش نیست که کلا چه کارهایی داشته و چه کارهایی را میخواست انجام دهد. او از آن جا که می‌داند درصورت رای آوردن احتمالی در انتخابات ssc کارهای او چند برابر می‌شود ، تصمیم گرفت برای مدیریت کارهایش با اصغردلاک مشورت کند. بعد از تفکر و مشورت فراوان با اصغردلاک به این نتیجه رسید که نیاز به یک داده ساختار دارد که برای مدیریت کارهایش از آن استفاده کند ، طوری که آخرین کاری که داشته یا اولین کاری که داشته و هنوز انجامش نداده یا کارهایی که برایش بیشتر از یک کار مشخص اولویت داشته را بتواند دریافت کند. از آن جایی که خیلی از دوستان اکبردلاک درگیر این مشکل هستند ، او تصمیم دارد که داده ساختاری را پیاده‌سازی کند که دوستانش نیز در عوض رای دادن به او در انتخابات (!) از آن استفاده کنند. با توجه به این که دوستان اکبر برای ذخیره کردن کارهایشان کلاس مخصوص به خودشان را دارند ، او باید داده ساختارش را به صورت generic پیاده سازی کند ، اما او به دلیل مسئولیت های فراوانی که دارد ، وقت انجام این کار را ندارد و شما باید داده ساختار را به شکل زیر براش پیاده سازی کنید. به طور دقیق نام داده‌ساختار و تعریف آن این‌گونه است:\\
\begin{tcolorbox}[boxrule=0pt]
	\begin{latin}
  	  \large{
  	  	public class AkbarWorks<T extends Comparable> \{\}
		}
	\end{latin}
\end{tcolorbox}
این کلاس باید عملیات زیر را انجام دهد که تعریف دقیق توابع متناظر آورده شده‌است.
\begin{itemize}
\item یک عنصر به مجموعه عناصری که تاکنون وجود دارند اضافه شود.
\begin{tcolorbox}[boxrule=0pt]
	\begin{latin}
  	  \large{
  	  	public void add (T item);
		}
	\end{latin}
\end{tcolorbox}
\item کوچکترین عنصر را حذف و آنرا خروجی دهد.
\begin{tcolorbox}[boxrule=0pt]
	\begin{latin}
  	  \large{
  	  	public T getMin () throws IllegalStateException;
		}
	\end{latin}
\end{tcolorbox}
\item عنصری که دیرتر از سایر عناصر اضافه شده است را خروجی دهد. اگر مقدار remove برابر با true بود، این عنصر را حذف هم بکند‫.‬
\begin{tcolorbox}[boxrule=0pt]
	\begin{latin}
  	  \large{
  	  	public T getLast (boolean remove) throws IllegalStateException;
		}
	\end{latin}
\end{tcolorbox}
\item عنصری که زودتر از سایر عناصر اضافه شده است را خروجی دهد. اگر مقدار remove برابر با true بود این عنصر را حذف هم بکند‫.
\begin{tcolorbox}[boxrule=0pt]
	\begin{latin}
  	  \large{
  	  	public T getFirst (boolean remove) throws IllegalStateException;
		}
	\end{latin}
\end{tcolorbox}‬
\item همه عناصری که کوچکتر از عنصر element هستند را خروجی دهد. اگر مقدار remove برابر با true بود این عناصر را حذف هم بکند‫.‬
\begin{tcolorbox}[boxrule=0pt]
	\begin{latin}
  	  \large{
  	  	public Comparable[ ] getLess (T element, bool remove);
		}
	\end{latin}
\end{tcolorbox}
\item ‌n عنصری که اخیرا حذف شده‌اند را (به ترتیب از اخیر ترین) خروجی دهد. (برای مقاصد بازیابی اطلاعات!) اگر \lr{n} از تعداد کل عناصر حذف‌شده بیشتر بود، تمام عناصر حذف‌شده را خروجی دهد.
\begin{tcolorbox}[boxrule=0pt]
	\begin{latin}
  	  \large{
  	  	public Comparable[ ] getRecentlyRemoved (int n);
		}
	\end{latin}
\end{tcolorbox}
\end{itemize}
نکات:
\begin{itemize}
\item در توابع getMin ، getLast و getFirst اگر هیچ عنصری وجود نداشته باشند باید یک استثنا از نوع IllegalStateException پرتاب شود.

\item توابع getMin ، getLast ، getFirst و getFirst روی عناصر موجود کار می‌کنند و اگر عنصری حذف شده باشد، این توابع آن عنصر را بررسی نمیکنند.
\item پیاده‌سازی داخلی و سایر متغیرها یا توابع کلاس اختیاری است.

\end{itemize}
\newpage




\section{شطرنج گرافیکی}


در تمرین سری قبل، مجبور شدید که برای این که جلوی خورده شدن خودتان توسط آدم‌خواران را بگیرید، برای آن‌ها یک بازی شطرنج طراحی کنید. آن‌ها بعد از مدتی بازی با آن شطرنج، به دلیل رابط کاربری متنی از آن خسته شدند و حال از شما خواسته‌اند که برای این که شما را نخورند، یک بازی شطرنج گرافیکی برای آنان تهیه کنید. به همین دلیل شما هم مجبور هستید که بازی شطرنج خود را به شکل گرافیکی تبدیل کنید و امیدوار باشید که بازی طراحی شده توسط شما آن‌قدر جذاب باشد که آنان به قدری مشغول بازی بشوند که بعد از خسته شدن از بازی، به قدری دچار ضعف بشوند که امکان خوردن شما برایشان مهیا نباشد!




همان طور که از توضیحات بند قبل مشخص است، هدف در این سؤال طراحی یک بازی شطرنج گرافیکی است.
در این تمرین، یکسری موارد اجباری و یکسری موارد اختیاری برای پیاده سازی وجود دارد. موارد اختیاری باعث کسب نمره اضافی علاوه بر نمره کامل سؤال می‌شوند.




\begin{itemize}[label = {$\blacksquare$}]

\item
توجه: برای راحتی کار، در صورتی که کدی که برای تمرین قبل زدید از کیفیت خوبی برخوردار باشد، می‌توانید از بخش‌هایی زیادی از آن کد برای این سؤال هم استفاده کنید ولی لزومی به این کار ندارید و اگر هم بخواهید، می‌توانید کل کد برنامه را دوباره بنویسید.

\setcounter{secnumdepth}{1}

\newpage
\subsection{بخش اجباری (30 نمره)}
\begin{enumerate}


\item
صفحه لاگین (ساختن و ورود به اکانت) – 5 نمره

امکاناتی که باید در صفحه مربوط به لاگین و ساخت اکانت موجود باشند:
\begin{itemize}[label = $\circ$]
\item
حذف اکانت (1 نمره)

\item
ساخت اکانت (1 نمره)

\item
لاگین (1 نمره)

\item
تغییر رمز عبور (1 نمره)

\item
خروج از بازی (1 نمره)

\end{itemize}

\item
منوی اصلی (5 نمره)

\begin{itemize}[label = $\circ$]
\item
نیوگیم

\begin{itemize}[label = $\Leftarrow$]
\item
در ابتدای ورود به بازی باید \lr{turn limit} و تعداد دفعات مجاز \lr{Undo} پرسیده شود؛ منظور از \lr{turn limit} چیست، در داک تمرین 2 توضیح داده شده است (2 نمره)
\end{itemize}



\item
اسکوربورد که به همان ترتیب گفته شده در تمرین 2 مرتب بشود و همه موارد تأثیرگذار در مرتب‌سازی (برد و باخت و...) برای هر بازیکن به شکل جدول یا هر شکل مناسب دیگری نمایش داده شوند. (2 نمره)

\item
قابلیت لاگ اوت (1 نمره)

\begin{itemize}[label = $\Leftarrow$]
\item
بعد از لاگ اوت کردن، طبیعتاً باید به منوی قبلی یعنی منوی لاگین و ثبت نام برگردید.
\end{itemize}

\end{itemize}

\item
خود بازی (20 نمره)

\begin{itemize}[label = $\circ$]

\item
همه موارد مربوط به بازی که در داک تمرین 2 گفته شده‌اند، می‌بایستی پیاده سازی بشوند. از جمله حرکت درست مهره‌ها مطابق قوانین شطرنج، انتخاب مهره‌ها، زدن مهره‌ها، نمایش درست صفحه شطرنج با اسامی درست و واقعی خانه‌ها یعنی ردیف‌های افقی اعداد 1 تا 8 و ستون‌های عمودی حروف A تا H انگلیسی. (10 نمره)

\item
وجود تصویر مناسب و متفاوت برای هر کدام از مهره‌ها؛ یعنی مهره‌ها فقط به صورت یک حرف الفبا نظیر Q و... روی صفحه نباشند و شکل مناسب برای آنان در نظر گرفته بشود. (2 نمره)

\item
امکان انصراف از بازی به‌وسیله یک \lr{Button} یا هر شکل دیگری که علاقه دارید. (1 نمره)

\item
پیاده سازی درست برد و باخت (1 نمره)

\item
امکان \lr{Undo} کردن به یک حرکت قبل و مشخص کردن تعداد \lr{Undo} های مجاز در ابتدای بازی و بازگشت به حالت قبل با انیمیشن (2 نمره)

\item
قرار دادن یک \lr{Button} یا هر شکل دیگری برای پایان نوبت (1 نمره)

\item
پیام‌های مناسب برای هر رویداد (مثلاً عدم امکان حرکت به یک نقطه – برد و اتمام بازی – تغییر نوبت و...) (2 نمره)

\item
داشتن انیمیشن برای حرکت مهره‌ها (1 نمره)

\end{itemize}

\end{enumerate}


\subsection{بخش امتیازی (15 نمره)}

هر مورد که پیاده سازی شود، 3 نمره خواهد داشت. بنابراین نیازی به پیاده سازی تمامی موارد نیست و پس از رسیدن به سقف 15 نمره امتیازی، نمره بیش‌تری برای سایر بخش‌ها داده نمی‌شود.

\begin{enumerate}


\item
امکان زمان‌دار کردن بازی. هر فرد یک زمان‌سنج جدا دارد که با به اتمام رسیدن زمان یک نفر در نوبت وی، آن فرد بازنده شده و فرد دیگر برنده می شود. زمان‌سنج هر نفر تنها در نوبت خود وی کار می‌کند.


\item
 پیاده سازی قابلیت \lr{Drag and Drop }برای مهره‌ها. توجه کنید که در صورت پیاده سازی این مهره، طبیعتاً مسئله \lr{Select} کردن مهره به شکلی شبیه تمرین 2 موضوعیت ندارد و در صورت پیاده سازی \lr{Drag and Drop} نمره آن داده می‌شود. همچنین در این حالت نیازی به پیاده سازی انیمیشن برای حرکت مهره‌ها (به جز در مورد \lr{Undo}) نیست. 
 چون مهره‌ها با \lr{Drag} شدن توسط موس حرکت خواهند کرد.

\newpage
\item
پیاده سازی قابلیت‌های قلعه کردن و آن پاسان. (هر مورد 5.1 نمره)


\begin{itemize}[label = $\Leftarrow$]

\item
\href{https://en.wikipedia.org/wiki/Castling}{\textcolor{blue}{لینک توضیحات قلعه کردن در ویکی‌پدیا}}

\item
\href{https://en.wikipedia.org/wiki/En_passant}{\textcolor{blue}{لینک توضیحات آن پاسان در ویکی‌پدیا}}


\end{itemize}


\begin{itemize}[label = $\blacksquare$]

\item
توجه:
در مورد قلعه کردن،‌ نیازی به رعایت شرط کیش نبودن شاه که در لینک‌ها ذکر شده نیست. همچنین در صورتی که بخش امتیازی مربوط به کیش و مات را پیاده سازی نکنید، نیازی به رعایت این که شاه بعد از حرکت در وضعیت کیش قرار نگیرد هم نیست. در صورتی که کیش و مات را به عنوان بخش امتیازی پیاده سازی کنید، باید این موضوع که شاه بعد از این حرکت در وضعیت کیش قرار نگیرد هم بررسی شود. سایر شرایط باید رعایت شوند.


\end{itemize}

\item
نگه‌داری لیست کامل حرکت‌ها و توانایی بازگشت به هر کدام از آن‌ها و ادامه دادن بازی از آنجا یعنی به نوعی قابلیت Undo کردن به هر حرکت از حرکات قبلی وجود داشته باشد. طبیعتاً تعداد دفعات انجام این کار، همانند شکل ساده Undo محدودیت دارد. در صورت پیاده سازی این قابلیت، منطقاً نیازی به پیاده سازی Undo یک حرکتی به صورت جداگانه نیست؛ چون پیاده سازی این مورد، به خودی خود این موضوع را هم شامل می‌شود.

\item
پخش موسیقی و صداگذاری

این موضوع دو جنبه دارد:
\begin{itemize}[label = $\Leftarrow$]

\item
موسیقی پس زمینه در کل بازی


\item
صداگذاری برای حرکت مهره‌ها و کلیک روی دکمه‌ها


\end{itemize}

 موسیقی پس زمینه، 5.1 نمره و صداگذاری هم 5.1 نمره دارد.
 

\newpage
\item
پیاده سازی کیش و مات. طبیعتاً در صورت پیاده سازی این موضوع، مسئله زده شدن شاه که برای سادگی در داک تمرین 2 مطرح شده بود، دیگر مطرح نیست و نیازی به پیاده سازی زده شدن شاه که در شطرنج واقعی وجود ندارد نیز نخواهد بود.

\begin{itemize}[label = $\Leftarrow$]

\item
\href{https://en.wikipedia.org/wiki/Check_(chess)}{\textcolor{blue}{لینک توضیحات کیش کردن در ویکی‌پدیا}}

\item
\href{https://en.wikipedia.org/wiki/Checkmate}{\textcolor{blue}{لینک توضیحات کیش و مات در ویکی‌پدیا}}


\end{itemize}

\item
نمایش حرکت‌های موجود برای هر مهره با انتخاب آن؛ یعنی مثلاً رنگ خانه‌هایی که این مهره می‌تواند به آن‌ها برود، کمی تغییر کند و از بقیه صفحه متمایز شود)

\item
تبدیل سرباز به مهره با ارزش تر در صورت رسیدن به انتهای زمین

\begin{itemize}[label = $\Leftarrow$]

\item
در صورتی که فقط امکان تبدیل به یک نوع مهره باشد 5.2 نمره به شما تعلق می‌گیرد.



\item
در صورتی که امکان انتخاب میان وزیر و رخ و اسب و فیل وجود داشته باشد 3 نمره به شما تعلق می‌گیرد.

\end{itemize}

\item
زیبایی بازی




\end{enumerate}

\setcounter{secnumdepth}{6}

\end{itemize}

\newpage


\section{فضانوردی}


در این تمرین قرار است حالت ساده‌ای از بازی \lr{space invaders} را پیاده‌سازی کنید. برای آشنایی با این بازی میتوانید به این 
\href{https://www.andoverpatio.co.uk/21/space-invaders/}{\textcolor{blue}{صفحه}}
 مراجعه کنید. تفاوت بازی اصلی با بازی که شما باید پیاده‌سازی کنید، این است که نیازی به پیاده‌سازی موانع نیست و همچنین نیازی نیست که هدف‌ها تیراندازی کنند و یا به چپ و راست حرکت کنند (این قسمت جزو موارد امتیازی محسوب میشود).در صورت رسیدن هدفها به پایین صفحه (جایی که سفینه‌ی هدف در آن قرار دارد)، بازی تمام میشود و پیام متناسب با آن نمایش داده میشود.

\subsection*{بخش اجباری (۳۰ نمره)}
\begin{enumerate}
\item جابه‌جا شدن سفینه‌ی بازیکن به چپ و راست با استفاده از کلیدهای جهت (۳ نمره)
\item تیراندازی توسط سفینه‌ی بازیکن با استفاده از یکی از کلیدهای کیبورد به دلخواه (۳ نمره)
\item حرکت کردن تیرها با استفاده از انیمیشن (۴ نمره)
\item قرار دادن هدف‌ها در ردیف‌های متفاوت و اینکه سفینه‌های هر ردیف رنگ متفاوتی داشته باشند. (۳ نمره)
\item جابه‌جا شدن هدف‌ها پس از بازه‌ی زمانی مشخص به یک ردیف پایین‌تر (۳ نمره)
\item نابودشدن سفینه‌های هدف در صورت برخورد تیر سفینه‌ی بازیکن به آن‌ها (۳ نمره)
\item درنظر گرفتن امتیاز برای هر بازی و نشان دادن آن در صفحه‌ی بازی (امتیاز هر بازی برابر با تعداد هدف‌هایی است که بازیکن به آنها شلیک و آنها را نابود میکند) (۴ نمره)
\item پیاده‌سازی منوی ساده برای ورود به بازی و خروج از آن با استفاده از کلیک‌کردن بر روی گزینه‌های آن (۳ نمره)
\item در نظرگرفتن حساب کاربری برای بازیکن‌های متفاوت و قراردادن نام کاربری برای هریک از آن‌ها (درصورتی که نام کاربری تکراری بود، پیام مناسب نشان داده شود) و امکان تغییر نام کاربری برای هر بازیکن و قرار دادن گزینه‌ی آن در منوی بازی (۴ نمره)
\end{enumerate}
\subsection*{بخش امتیازی (۱۵ نمره)}
هر مورد که پیاده سازی شود، ۳ نمره خواهد داشت. بنابراین نیازی به پیاده سازی تمامی موارد نیست و پس از رسیدن به سقف ۱۵ نمره امتیازی، نمره بیش‌تری برای سایر بخش‌ها داده نمی‌شود.


\begin{enumerate}
\item تیراندازی توسط هدف‌ها به صورت رندوم که در صورت برخورد به سفینه‌ي هدف، بازی تمام می‌شود (در این حالت نیز پیام مناسب برای پایان بازی باید نمایش داده شود).
\item جابه‌جا شدن سفینه‌های هدف به چپ و راست (همانند بازی اصلی)
\item پیاده‌سازی جدول امتیازات و قرار دادن گزینه‌ی آن در منوی بازی
\item بعد از اینکه سفینه‌های هدف یک ردیف پایین‌تر آمدند، یک ردیف از بالا به این سفینه‌ها اضافه شود.همچنین مدت زمانی که طول میکشد تا یک ردیف به هدف‌ها اصافه شود، هر بار کاهش پیدا کند تا به مقدار معینی برسد.بیشترین و کمترین این مقدار زمانی را در ابتدای بازی باید از بازیکن ورودی بگیرد.
\item برای سفینه‌ی بازیکن یک قدرت مخصوص پیاده‌سازی کنید.این قدرت به صورت یک هدف در فضای خالی بین سفینه‌ی بازیکن و سفینه‌های هدف قرار میگیرد که بازیکن با تیراندازی به آن می‌تواند آن قدرت را دریافت کند (اگر فضای خالی وجود نداشت ظاهر نمیشود). توجه کنید که پس از دریافت آن، برای مدت زمان معینی میتوان از آن استفاده کرد و سپس از بین میرود.این قدرت به همراه اهداف دیگر در بازه‌ی زمانی مشخصی پایین می‌آید و اگر تا ردیف آخر گرفته نشود از بین میرود.از بین دو قدرت زیر، یکی را به دلخواه میتوانید پیاده‌سازی کنید :
\begin{itemize}[label = $\circ$]
\item مسلسل : سرعت تیر‌زدن را چند‌برابر می‌کند که بعد از مدت زمان مشخصی این قدرت از دست می‌رود.
\item بمب : قابلیت پرتاب بمب به بازیکن می‌دهد که در صورت شلیک آن، چند هدف کنار هم نابود می‌شوند.
\end{itemize}
\item قراردادن تعدادی مانع بین بازیکن و سفینه‌های هدف که در صورت برخورد هرکدام از تیرهای سفینه‌های هدف یا سفینه‌ی بازیکن با یکی از موانع، آن مانع از بین می‌رود.
\item صداگذاری برای بازی (مثل صدای شلیک تیر) و همچنین موسیقی پس زمینه
\item در کل زیباسازی هر چه بیشتر بازی با استفاده از نکته‌های گرافیکی
\end{enumerate}


\newpage

\begin{comment}
\section{شبکه شتاب}

در این تمرین قرار است یک شبکه‌ی بانکداری را پیاده سازی کنید:


این شبکه متشکل از سه بخش است:
\begin{enumerate}
\item یک سرور DNS (برای اطلاع بیشتر درباره‌ی ‌DNS ها می‌توانید از این 
\href{https://www.cloudflare.com/learning/dns/what-is-dns/}{\textcolor{blue}{لینک}}
 استفاده کنید.)
\item تعدادی بانک که هر کدام یک سرور مختص به خود دارند.
\item تعدادی عابربانک که هر کدام با سرور بانک خود در ارتباط هستند.
\end{enumerate}

\begin{itemize}


\item \textbf{سرور بانک:}
در هر بانک تعدادی حساب وجود دارد. هر حساب توسط شماره‌ حساب آن، که یک عدد یکتاست، شناخته می‌شود .
همچنین هر حساب مقداری موجودی دارد. تنها عملیاتی که بر روی یک حساب انجام می‌شود، واریز یا برداشت وجه است.
در هر عملیات واریز، اگر شماره حساب مقصد موجود نبود، ابتدا یک حساب جدید با آن شماره حساب و موجودی صفر ساخته شده،سپس  واریز صورت می‌گیرد. 
همچنین هر عملیات برداشت در صورتی انجام می‌گیرد که آن شماره حساب موجود باشد و مقدار برداشتی حداکثر برابر موجودی حساب باشد.در غیر این صورت این عملیات نادیده گرفته می‌شود.
در ابتدای کار هیچ حسابی در بانک ها وجود ندارد و حساب ها تنها توسط عملیات واریز ساخته می‌شوند.
همچنین حساب‌ها هیچ گاه از‌ بین نمی‌روند.
زمانی که یک بانک جدید ایجاد می‌شود، ابتدا شماره‌ی port خود را به همراه نام بانک برای سرور DNS می‌فرستد.
سپس منتظر اتصال عابر بانک ها می‌‌شود. در فلان صفحه نحوه ایجاد بانک جدید توضیح داده می شود.

\item \textbf{عابر بانک:}
زمانی که یک عابر بانک جدید ایجاد می‌شود، ابتدا نام بانک خود را برای سرور DNS می‌فرستد.
سپس سرور DNS در پاسخ، شماره‌ی port سرور آن بانک را می‌فرستد.در نهایت عابر بانک با استفاده از آن port به سرور بانک خود متصل می‌شود و از آن پس تراکنش‌ها را برای سرور بانک خود ارسال می‌کند تا بررسی و اعمال شوند. در فلان صفحه نحوه ایجاد عابر بانک جدید توضیح داده می شود.

\item \textbf{سرور DNS :}
کار آن این است که به ازای هر بانک، شماره‌ی port آن را ذخیره کند.
خود این سرور نیز شماره‌ی port ثابتی دارد که از ابتدا مشخص است و هم بانک‌ها و هم عابر بانک ها از آن اطلاع دارند و با استفاده از آن با سرور DNS ارتباط برقرار می‌کنند.
\end{itemize}




\subsection*{پیاده سازی}
برای این منظور کد خام و تست های این تمرین در اختیار شما قرار می‌‌گیرد که شما می‌بایست کد خام را به گونه ای تکمیل کنید که تست ها را پاس کنید.
تست ها را \lr{JUnit } اجرا کنید.
در نهایت کافی است فولدر src کامل شده را آپلود کنید.

\begin{itemize}
\item \textbf{کلاس DNS :}\\

این کلاس سرور DNS را پیاده‌ سازی می‌کند.
\begin{itemize}[label = {}]
\item	
\begin{tcolorbox}[boxrule=0pt]
	\begin{latin}
  	  \large{
  	  	public DNS(int dnsPort)
		}
	\end{latin}
\end{tcolorbox}

کانستراکتور کلاس DNS است. dnsPort هم شماره‌ی port ‌سرور است.
بعد از ایجاد تنها شی  این کلاس ، باید سرور DNS فعال شده و بر روی پورت dnsPort منتظر اتصال بانک‌ها وعابر بانک‌ها شود.

\item	
\begin{tcolorbox}[boxrule=0pt]
	\begin{latin}
  	  \large{
  	  	public int getBankServerPort(String bankName)
		}
	\end{latin}
\end{tcolorbox}

باید port سرور بانک با اسم bankName را برگرداند.
این تابع صرفا از قسمت Unit Test صدا زده خواهد شد و هیچ کدام از کلاس ها و توابعی که شما پیاده سازی می کنید حق صدا زدن این تابع را ندارند.
\end{itemize}

\item \textbf{کلاس BankServer :}\\

این کلاس، بانک ها را پیاده‌ سازی می‌کند.
\begin{itemize}[label = {}]
\item	
\begin{tcolorbox}[boxrule=0pt]
	\begin{latin}
  	  \large{
  	  	public BankServer(String bankName, int dnsPort)
		}
	\end{latin}
\end{tcolorbox}

کانستراکتور بانک است. برای ایجاد یک بانک جدید با نام bankName و port با مقدار dnsPort این تابع صدا می‌شود. هر بانک تنها یک سرور دارد و بعد از صدا شدن کانستراکتورش، باید با استفاده از dnsPort به سرور DNS متصل شود و شماره‌ی ‌port خود را به همراه اسم بانک اعلام کند. پس از آن بانک ساخته شده منتظر اتصال عابر بانک ها می‌ماند و بعد از اتصال هر عابر بانک، تراکنش های آن را انجام می‌دهد.

\item	
\begin{tcolorbox}[boxrule=0pt]
	\begin{latin}
  	  \large{
  	  	public int getBalance(int userId)
		}
	\end{latin}
\end{tcolorbox}

موجودی حساب با شماره‌ی userId را برمی‌گرداند.
این تابع صرفا از قسمت Unit Test صدا زده خواهد شد و هیچ کدام از کلاس ها و توابعی که شما پیاده سازی می کنید حق صدا زدن این تابع را ندارند.

\item	
\begin{tcolorbox}[boxrule=0pt]
	\begin{latin}
  	  \large{
  	  	public int getNumberOfConnectedClients()
		}
	\end{latin}
\end{tcolorbox}

تعداد عابر بانک های متصل شده یک بانک را برمی گرداند.
این تابع صرفا از قسمت Unit Test صدا زده خواهد شد و هیچ کدام از کلاس ها و توابعی که شما پیاده سازی می کنید حق صدا زدن این تابع را ندارند.
\end{itemize}

\item \textbf{کلاس BankClient :}\\

این کلاس یک عابر بانک را پیاده‌ سازی می‌کند.
\begin{itemize}[label = {}]
\item	
\begin{tcolorbox}[boxrule=0pt]
	\begin{latin}
  	  \large{
  	  	public BankClient(String bankName, int dnsPort)
		}
	\end{latin}
\end{tcolorbox}

کانستراکتور آن است. برای ایجاد یک عابر بانک جدید برای بانکی با نام bankName یک شی از این کلاس ساخته می شود. هر بانک می‌تواند چند عابر بانک داشته باشد. بعد از صدا شدن کانستراکتور، عابر بانک باید با استفاده از dnsPort به سرور DNS متصل شود و بعد از ارسال اسم بانک، شماره‌ی ‌port بانک خود را دریافت کند سپس به سرور بانک خود متصل شود.



\item
\begin{tcolorbox}[boxrule=0pt]
	\begin{latin}
  	  \large{
  	  	public void sendTransaction(int userId, int amount)
		}
	\end{latin}
\end{tcolorbox}

توسط این تابع، یک تراکنش برای شماره‌ حساب userId به سرور فرستاده می شود.
اگر amount نامنفی بود، مقدار ‌amount واریز می‌شود.
در غیر این‌ صورت مقدار |amount| واحد برداشت می‌شود.
دقت کنید که پردازش این عملیات باید در سرور اتفاق بیفتد و کلاینت صرفا درخواست را می فرستد.


\item
\begin{tcolorbox}[boxrule=0pt]
	\begin{latin}
  	  \large{
  	  	public void sendAllTransactions(String fileName, int timeBetweenTransactions)
		}
	\end{latin}
\end{tcolorbox}

این تابع باید از فایل fileName لیست تراکنش‌ها را بخواند و آن ها را به ترتیب اجرا کند. بین اجرای هر دو تراکنش متوالی هم باید حداقل timeBetweenTransactions میلی‌ثانیه فاصله باشد. در هر خط از فایلfileName ،مشخصات یک تراکنش آمده است.
هر تراکنش با دو عدد نمایش داده شده است. عدد اول شماره‌ی حساب و عدد دوم مقدار تراکنش است. همانند بالا علامت عدد دوم نمایانگر نوع تراکنش است.
\end{itemize}
\end{itemize}


\subsection*{قوانین}
\begin{enumerate}
\item توجه کنید تمام تراکنش ها در سرور بانک صورت می‌گیرند و عابر بانک فقط اطلاعات تراکنش را برای سرور بانک می‌فرستد.(هیچ قسمت از پردازش نباید در کلاینت اتفاق بیفتد.)

\item دقت کنید که برای آنکه بتوانید تست ها را پاس کنید لازم است بعضی توابع (یا constructor) ها را به صورت Blocking پیاده سازی کنید و بعضی دیگر را به صورت Non-Blocking.

\begin{itemize}

\item  تابع Blocking : تا زمانی که عملیات خواسته شده به صورت کامل انجام نشده است از تابع خارج نمی شود و در نتیجه برنامه اصلی(caller) متوقف می ماند.

\item تابع non-blocking : عملیات خواسته شده را در یک ترد دیگر آغاز می کند و از تابع خارج شده و برنامه اصلی(caller) به اجرایش ادامه می دهد.
\end{itemize}
\item لازم است که در ابتدای توابعی که پیاده سازی میکنید نوع آن تابع ( Blocking / Non-Blocking ) را ذکر کنید.

\item ممکن است تست ها در تحویل حضوری تغییرات مقداری داشته باشند. اما ساختار و نمره تست ها دقیقا به همین صورت باقی می مانند.

\item سقف نمره شما را تست هایی که در تحویل حضوری پاس میکنید مشخص میکنند و در صورت عدم کامنت کردن نوع توابع و یا نداشتن توضیح کافی برای دلیل انتخابشان( مثلا دلیل انتخاب Blocking ) مقداری از نمره را از دست می دهید. (برای دلیل کافی است بگوید که مثلا اگر فلان تایع NonBlocking میشد فلان اتفاق می افتاد و در نتیجه فلان تست پاس نمی شد). در صورتی که هر دو نوع قابل انتخاب باشند و تست ها را پاس کنند، به صلاحدید خودتان یکی را انتخاب کنید.

\item در صورتی که کلاس های شما علاوه بر بستر شبکه از راه های دیگری با هم ارتباط داشته باشند ( مثلا ارتباط از طریق صدا زدن تابع های همدیگر و یا ارتباط از طریق اشتراک گذاری فایل یا … ) نمره تان 0 خواهد شد.

\item هدف از این تمرین پیاده سازی یک شبکه سریع و ایمن است. پس استفاده به جا از Thread و همچنین دقت به عدم تداخل ترد های مختلف بخش اصلی تمرین است و سعی شده تا حد توان در تست های داده شده این موارد چک شوند.

\item برای قسمت هایی که در داک محدودیتی گفته نشده شما مجاز هستید طبق صلاحدید خودتان تصمیم بگیرید.
\end{enumerate}
\href{https://drive.google.com/file/d/1FKnRqBx1y2hWrEFyPeT2nMLZPFuENcHt/view?usp=sharing}{\textcolor{blue}{فایل خام تمرین}}
\end{comment}
\end{document}









